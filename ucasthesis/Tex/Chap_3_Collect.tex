\chapter{自动化移动应用HTTPS流量标记方法}\label{chap:collect}
在课题研究过程中,需要大量的数据支持。需要在大量的数据上训练移动应用流量识别模型以及验证模型的效果。然而,当前所公布的开源数据较少,数据量有限,分类的粒度较为粗糙。本章详细地描述了移动应用HTTPS流量数据地采集和处理方法,构建了一个50个不同类型的移动应用的流量数据集,规模为20万条流样本。

\section{公开流量数据集}
本章首先本文调研了当前已经公开的流量数据集,总结这些公开数据集的来源、规模、协议、用途以及获取方式。现有的公开并被多项研究工作使用的数据集如下:
\begin{itemize}
    \item \citep{draper2016characterization}发布了VPN-nonVPN dataset (ISCXVPN2016)数据集\footnote{https://www.unb.ca/cic/datasets/vpn.html},其中包括7种常规加密流量和7种协议封装流量。该项工作捕获了一个常规会话和一个通过VPN的会话,因此,共有14种流量类别:VOIP,VPN-VOIP,P2P,VPN-P2P等。数据内容包括:
    请见表~\ref{tab:sample}。
    \begin{table}[!htbp]
        \bicaption{ISCXVPN2016 数据集。}{ISCXVPN2016 Dataset.}
        \label{tab:sample}
        \centering
        \footnotesize% fontsize
        \setlength{\tabcolsep}{4pt}% column separation
        \renewcommand{\arraystretch}{1.2}%row space 
        \begin{tabular}{lc}
            \hline
            类型 & 内容 \\
            \hline
            Traffic & Content \\
            Web Browsing & Firefox and Chrome \\
            Email & SMPTS, POP3S and IMAPS \\
            Chat & ICQ, AIM, Skype, Facebook and Hangouts \\
            Streaming & Vimeo and Youtube \\
            File Transfer & Skype, FTPS and SFTP using Filezilla and an external service \\
            VoIP & Facebook, Skype and Hangouts voice calls (1h duration) \\
            P2P & uTorrent and Transmission (Bittorrent)\\
            \hline
        \end{tabular}
    \end{table}
    
    \item Intrusion detection evaluation dataset (ISCXIDS2012) \footnote{https://www.unb.ca/cic/datasets/ids.html}:该数据集包括标记的网络跟踪,包括pcap格式的完整数据包有效负载,以及相关的配置文件供研究人员公开使用。如表:\ref{tab:ISCXIDS2012}所示,由7天的网络活动(正常和恶意)组成:
     \begin{table}[!htbp]
        \bicaption{ISCXIDS2012数据集。}{ISCXIDS2012 Dataset.}
        \label{tab:ISCXIDS2012}
        \centering
        \footnotesize% fontsize
        \setlength{\tabcolsep}{4pt}% column separation
        \renewcommand{\arraystretch}{1.2}%row space 
        \begin{tabular}{lcc}
        \hline
        \textbf{日期} &  \textbf{描述} & \textbf{数据规模(GB)}\\
        \hline
        11/6/2010 & Normal Activity. No malicious activity & 16.1\\
        12/6/2010 & Normal Activity. No malicious activity & 4.22\\
        13/6/2010 & Infiltrating the network from inside + Normal Activity & 3.95\\
        14/6/2010 & HTTP Denial of Service + Normal Activity & 6.85\\
        15/6/2010 & Distributed Denial of Service using an IRC Botnet & 23.4\\
        16/6/2010 & Normal Activity. No malicious activity & 17.6\\
        17/6/2010 & Brute Force SSH + Normal Activity & 12.3\\
        \hline
        \end{tabular}
    \end{table}
    该数据集合主要针对恶意流量的分析,包含了HTTP,SSH等协议诸如DDOS,Brute Force等恶意行为,因此该数据集在本文研究的任务重合度有限。
    
    \item \citep{bujlow2015independent}提出数据集Independent Comparison of Popular DPI Tools for Traffic Classification" dataset\footnote{https://cba.upc.edu/monitoring/traffic-classification},这个数据集包含767690个流,这些流占53.31GB的纯数据包数据。存在759720个流的应用程序名称(占所有流的98.96%),占数据量的51.93GB(97.41%)。数据集由一个pcap跟踪和一个INFO文件组成。INFO文件中的每一行对应于pcap跟踪中的流,并描述如下:
    \begin{lstlisting}
     flow_id + "#" + start_time + "#" + end_time + "#" + local_ip + "#" + remote_ip + "#" + local_port + "#" + remote_port + "#" + transport_protocol + "#" + operating_system + "#" + process_name + "#" + HTTP Url + "#" + HTTP Referer + "#" + HTTP Content-type +"#"
    \end{lstlisting}
    本数据集数据规模较大,但是数据是已经预处理,对于新的识别问题存在局限。
    
    \item DARPA99 traces\footnote{https://www.ll.mit.edu/r-d/datasets}:该ARPA99跟踪是来自MIT Lincoln实验室1999年的模拟网络的数据包。这些数据包中提供了所有网络数据,其标签是不同的攻击行为,因此该数据集不符合本文的研究场景。
    
    \item MAWILab\footnote{http://mawi.wide.ad.jp/mawi/}:\citep{mawilab}发布了MAWILab数据集,可帮助研究人员评估其交通异常检测方法。它由一组用于定位MAWI存档中交通异常的标签(样本点B和F)组成。使用基于图形的高级方法获得标签,该方法比较并组合了不同且独立的异常检测器。 数据集每天更新,以包括来自即将到来的应用程序和异常的新流量。该数据集是用于评估异常流量检测方法的,不符合本文的研究场景。
    
    \item NLANR AMP Data\footnote{https://labs.ripe.net/datarepository/data-sets/nlanr-amp-data}:此数据是NLANR研究小组收集的一组活动测量(ping /traceroute)。 数据由多达130个有利点的网格中的测量组成,并且这些测量在1998年至2006年之间进行。该数据可用于Internet的纵向研究。RIPE数据存储库中的可用数据是NLANR网站上可用的原始数据的重新制作版本。该数据集包括的是\emph{ping/traceroute}和相关的流量,不符合本文的研究场景所需。
    
    \item NIMS\footnote{https://projects.cs.dal.ca/projectx/Download.html}:在研究工作中\citep{alshammari2011can,alshammari2008investigating,alshammari2007flow}提出了NIMS数据集,该数据的标签为:\emph{TELNET, FTP, HTTP, DNS, lime, localForwarding, remoteForwarding, scp, sftp, x11, shell},是基于协议层的分类,且单条样本已经被预处理为22个字段的数值,主要为均值、方差、最大值、最小值等统计特征。该数据集涉及了多种加密协议的样本,在识别粒度上要比本文的识别任务粗糙,不符合本文场景。
    
    \item WITS: Waikato Internet Traffic Storage\footnote{https://wand.net.nz/wits/}:该数据集目前有33个不同的集合构成,数据规模较为庞大,且数据来源较为丰富,但是该数据集采集的是混合流量,样本缺少精细化的标记,因此难于用于训练精细化的流量识别方法,无法满足本文中的精细化识别场景的需求。
    
    
\end{itemize}


本课题的目的在于通过对大量的应用的HTTPS流量数据进行分析,得到一种可靠准确的分析方法,经过分析,以上提到的公开数据集在通信协议、数据规模、数据标签、数据格式或者使用场景等不满足本文针对HTTPS流量进行精细化识别的要求。为此需要构建新的数据集来支撑本文的研究工作。

\section{流量采集}
在本文章中介绍一种针对Android应用流量的自动化/半自动化的流量采集方法。数据采集的流程如图所示,如图:\ref{fig:data-collect}所示,该系统主要由APP爬虫、应用调度、App运行环境、事件注入和守护进程五个模块构成。
\begin{figure}[!htbp]
	\centering
	\includegraphics[width=0.80\textwidth]{Data-collect.pdf}
	\bicaption{数据采集}{Data collect}
	\label{fig:data-collect}
\end{figure}
为了便捷地从应用市场下载应用,首先设计了爬虫工具,从应用市场爬取移动应用的安装包。然后应用调度模块选取安装文件进行分发,在应用的运行环境(虚拟机和真机)完成安装,接着事件注入层通过一些列操作事件驱动应用程序自动运行,同时守护进程中的流量捕获模块(核心为Wireshark \footnote{https://www.wireshark.org/}和TcpDump\footnote{http://www.tcpdump.org})会开始抓取流量,在单个真实的设备和模拟器中都执行一个应用程序。 该应用由Android工具monkeyrunner\footnote{https://developer. android. com/studio/test/monkeyrunner/index.html}自动驱动。为了避免设备本身的影响,同时在不同设备上执行同一应用,并通过限制设备中其他应用的网络权限来确保没有应用在后台运行。一次捕获一个应用程序的流量,以确保采集的数据的标签即为当前的应用名。 为的了避免受到网络环境的影响,在一个月的不同时间段捕获了流量。

\section{预处理}
我们以流为单位进行研究和标记,我们在采集流量的时候,保存的同一个pcap文件是由多条流构成的。在流量被采集后,通过五元组信息---\textit{[源地址,源端口,目的地址,目的端口,传输协议]}将pcap文件进行切分,pcap文件中的包(packet)重组后形成新的pcap文件,被切分后,每一个pcap文件都是一条样本,本文中使用的抽取方法介绍如下。



\subsection{基于splitcap抽取流}
SplitCap\footnote{https://www.netresec.com/?page=SplitCap}是一个免费工具,旨在根据诸如IP地址,5元组或MAC地址等标准将捕获文件(PCAP文件)拆分为较小的文件。 可用于拆分/分组的标准是:
\begin{itemize}
    \item BSSID:根据WLAN BSSID分组的数据包
    \item 流:每个5元组的单向流量(传输协议,IP地址和端口号)分组在一起。
    \item 主机:将流量按IP地址(源和目标)分组到一个文件。 大多数数据包将以两个文件结尾。
    \item 主机对:基于IP对通信进行分组的流量。
    \item MAC地址:将流量按每个MAC地址分组到一个文件。 大多数数据包将以两个文件结尾。
    \item 会话:每个会话的数据包(双向流)被分组在一起。
    \item 时间:根据时间拆分。
    \item 数据包计数:根据数据包计数拆分。
\end{itemize}
本文中选择按照流的标准来切分流,具体的操作代码如下:
\begin{lstlisting}[language=sh]
foreach($f in gci 1_Pcap *.pcap)
{
    SplitCap -p 100000 -b 100000 -r $f.FullName -o 2_Session\AllLayers\$($f.BaseName)-ALL
    SplitCap -p 100000 -b 100000 -r $f.FullName -s flow -o 2_Session\AllLayers\$($f.BaseName)-ALL
    gci 2_Session\AllLayers\$($f.BaseName)-ALL | ?{$_.Length -eq 0} | del
}
\end{lstlisting}

\subsection{基于Scapy抽取流}
Scapy是一个Python程序,使用户能够发送,嗅探,剖析和伪造网络数据包。 此功能允许构建可以探测,扫描或攻击网络的工具。Scapy是功能强大的交互式数据包处理程序。 它能够伪造或解码各种协议的数据包,在线发送它们,捕获它们,匹配请求和答复等等。 Scapy可以轻松处理大多数经典任务,例如扫描,跟踪路由,探测,单元测试,攻击或网络发现。它可以替代\textit{hping,arpspoof,arp-sk,arping,p0f}甚至\textit{Nmap,tcpdump和tshark}的某些部分。

本文中利用Scapy按照五元组信息重组pcap文件,提取流,并将提取的流保存为pcap文件格式,这里附上提取代码\footnote{http://www.damaoguo.site/2020/03/06/ExtractFlowWithPython/}。

原始的流量数据经过处理以后,一个原始的pacp文件将会按照流拆分为新的pcap小文件,文件存储的目录结构如图:\ref{fig:file-tree}所示。
\begin{figure}[!htbp]
	\centering
	\includegraphics[width=0.80\textwidth]{File-tree.pdf}
	\bicaption{流数据文件存储目录}{Traffic Flow File Tree}
	\label{fig:file-tree}
\end{figure}



\section{数据集概览}
我们基于HTTPS协议创建流量数据集。共采集了旅行交通、社交、影音视听、时尚购物新闻资讯、居家生活、聊天、图书阅读、金融理财、实用工具、游戏共计11种类型,50个应用的流量,使用sliptcap切分后得到20万多万条流量样本。

\begin{longtable}{c|c|c|c}
    \bicaption{移动应用HTTPS流量数据集}{mobile apps HTTPS Traffic Dataset}\\
	\hline
	\textbf{应用名} & \textbf{开发商} & \textbf{应用类别} & \textbf{应用数量} \\
	\hline
	\endfirsthead
	\multicolumn{4}{c}%
        {\bfseries\small \tablename\ \thetable\ {续表。}}\\
	\hline
	\textbf{应用名} & \textbf{开发商} & \textbf{应用类别} & \textbf{应用数量} \\
	\hline
	\endhead
	\hline \multicolumn{4}{r}{\textit{续表见下页}}\\
	\endfoot
	\hline
	\endlastfoot
	百度地图 & Baidu.com & 旅行交通 & 6777\\
	\hline
	百度贴吧 & Baidu.com & 社交 & 3234\\
	\hline
	网易云音乐 & Netease.com & 影音视听 & 9888\\
	\hline
	爱奇艺 &  iQIYI & 影音视听 & 2634\\
	\hline
	京东 & JD & 购物 & 7956\\
	\hline
	今日头条 & ByteDance & 新闻资讯 & 5321\\
	\hline
	美团 & Meituan.com& 居家生活 & 12469\\
	\hline
	QQ & Tencent & 聊天 & 1246\\
	\hline
	QQ音乐 & Tencent & 影音视听 & 1155\\
	\hline
	QQ阅读 & Tencent & 图书阅读 & 1563\\
	\hline
	淘宝 & Taobao & 购物 & 3431\\
	\hline
	微博 & Sina & 社交 & 3097\\
	\hline
	携程 & CTRIP & 居家生活 & 2141\\
	\hline
	知乎 & Zhihu.com & 社交 & 2011\\
	\hline
	抖音 & Douyin.com & 社交 & 7441 \\
	\hline
	饿了么 & Ele.me & 居家生活 & 16053\\
	\hline
	国泰君安 & gtja.com & 金融理财 & 7734\\
	\hline
	QQ邮箱 & Tencent & 实用工具 & 4879\\
	\hline
	腾讯新闻 & Tencent & 新闻资讯 & 5679\\
	\hline
	支付宝 & Alipay.com & 金融理财 & 2301\\
	\hline
	阿里健康 & alihealth.cn & 居家生活 & 22904\\
	\hline
	安居客 & anjuke.com & 居家生活 & 2340\\
	\hline
	百词斩 & baicizhan.com & 实用工具 & 674\\
	\hline
	百合婚恋 &  baihe.com & 居家生活 & 2452\\
	\hline
	贝壳找房 & bj.ke.com & 居家生活 & 7520\\
	\hline
	当当阅读 & dangdang.com & 图书阅读 & 2588\\
	\hline
	钉钉 & dingtalk.com & 居家生活 & 3468\\
	\hline
	丁香 & dxy.cn & 居家生活 & 4654\\
	\hline
	豆瓣 & douban.com & 社交 & 3998\\
	\hline
	火山小视频 & huoshan.com & 影音视听 & 2924\\
	\hline
	Keep & www.gotokeep.com & 居家生活 & 9646\\
	\hline
	秒拍 & miaopai.com & 社交 & 340\\
	\hline
	中国南方航空 & China Southern Airlines & 旅行交通 & 7656\\
	\hline
	拼多多 & pinduoduo.com & 购物 & 2660\\
	\hline
	蜻蜓FM & 影音视听 & 社交 & 2312 \\
	\hline
	去哪儿 & qunar.com & 居家生活 & 3294\\
	\hline
	Soul & soulapp.cn & 聊天 & 4406\\
	\hline
	天涯 & tianya.cn & 聊天 & 1058\\
	\hline
	天眼查 & tianyancha.com & 实用工具 & 6146\\
	\hline
	同花顺 & 10jqka.com.cn & 金融理财 & 5928\\
	\hline
	王者荣耀 & Tencent & 游戏 & 2524\\
	\hline
	闲鱼 & 2.taobao.com & 购物 & 3226\\
	\hline
	小米运动 & huami.com & 居家生活 & 2364\\
	\hline
	新浪财经 & 金融理财.sina.com.cn & 金融理财 & 3608\\
	\hline
	央视新闻 & news.cctv.com & 社交 & 1534 \\
	\hline
	有道云笔记 & note.youdao.com & 实用工具 & 2396\\
	\hline
	掌上生活 & cmbchina.com & 金融理财 & 4838\\
	\hline
	直播吧 & zhibo8.cc & 实用工具 & 4368\\
	\hline
	中国国际航空 & 中国国际航空 & 旅行交通 & 3343\\
	\hline
	作业帮 & Zybang.com & 实用工具 & 4692\\
	\hline
	\hline
	\textbf{总计} & - & - & \textbf{236871}\\
	\hline
\end{longtable}

\section{小结}
本章提出了一个自动化的移动应用HTTPS流量采集标记方法,并实现了相关的系统,关键的技术分别是应用的自动执行、大规模流量生成和流量标记问题。通过本章的研究,为本文的研究工作提供了可靠的流量数据。