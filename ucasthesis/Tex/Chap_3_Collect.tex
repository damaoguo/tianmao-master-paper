\chapter{数据集和数据预处理}\label{chap:collect}
在课题研究过程中,需要大量的数据支持。需要在大量的数据上训练移动应用流量识别模型以及验证模型的效果。然而,当前所公布的开源数据较少,数据量有限,分类的粒度较为粗糙。


\section{流量采集}
在本文章中介绍一种针对Android应用流量的自动化/半自动化的流量采集方法。数据采集的流程如图所示,该系统主要由APP爬虫、应用调度、App运行环境、事件注入和守护进程五个模块构成。
\begin{figure}[!htbp]
	\centering
	\includegraphics[width=0.80\textwidth]{Data-collect.pdf}
	\bicaption{数据采集}{Data collect}
	\label{fig:data_collect}
\end{figure}

\section{预处理}
我们以流为单位进行研究和标记,离线的数据流结构如图,我们在采集流量的时候,保存的同一个pcap文件是由多条流构成的。
\begin{figure}[!htbp]
	\centering
	\includegraphics[width=0.80\textwidth]{flow_offline}
	\bicaption{离线流数据}{}
	\label{fig:flow_offline}
\end{figure}
\subsection{基于splitcap抽取流}
\subsection{基于Scapy抽取流}
\subsection{数据集概览}
我们基于HTTPS协议创建流量数据集。 数据集包含20个流行应用程序的100,000多个HTTPS流。 我们在多个真实的Android设备和Android模拟器中都执行一个应用程序。 该应用由Android工具monkeyrunner\cite{developers2015monkeyrunner}自动驱动。 我们会同时在不同设备上执行同一应用,并通过限制android设备中其他应用的权限来确保没有应用在后台运行。 我们一次捕获一个应用程序的流量,以确保它是事实。 为了避免受到网络环境的影响,我们在一个月的时间里,一直在手动和自动方式下通过Wireshark \cite{wu2018wireshark}和tcpdump \cite{jacobson2003tcpdump}来捕获流量。


共采集了旅行交通、社交、影音视听、时尚购物新闻资讯、居家生活、聊天、图书阅读、金融理财、实用工具、游戏共计11种类型,50个应用的流量,使用sliptcap切分后得到20万多万条流量样本。


\begin{longtable}{c|c|c|c}
    \bicaption{Android应用}{Android apps}\\
	\hline
	\textbf{应用名} & \textbf{开发商} & \textbf{应用类别} & \textbf{应用数量} \\
	\hline
	\endfirsthead
	\multicolumn{4}{c}%
        {\bfseries\small \tablename\ \thetable\ {续表。}}\\
	\hline
	\textbf{应用名} & \textbf{开发商} & \textbf{应用类别} & \textbf{应用数量} \\
	\hline
	\endhead
	\hline \multicolumn{4}{r}{\textit{续表见下页}}\\
	\endfoot
	\hline
	\endlastfoot
	百度地图 & Baidu.com & 旅行交通 & 6777\\
	\hline
	百度贴吧 & Baidu.com & 社交 & 3234\\
	\hline
	网易云音乐 & Netease.com & 影音视听 & 9888\\
	\hline
	爱奇艺 &  iQIYI & 影音视听 & 2634\\
	\hline
	京东 & JD & 购物 & 7956\\
	\hline
	今日头条 & ByteDance & 新闻资讯 & 5321\\
	\hline
	美团 & Meituan.com& 居家生活 & 12469\\
	\hline
	QQ & Tencent & 聊天 & 1246\\
	\hline
	QQ音乐 & Tencent & 影音视听 & 1155\\
	\hline
	QQ阅读 & Tencent & 图书阅读 & 1563\\
	\hline
	淘宝 & Taobao & 购物 & 3431\\
	\hline
	微博 & Sina & 社交 & 3097\\
	\hline
	携程 & CTRIP & 居家生活 & 2141\\
	\hline
	知乎 & Zhihu.com & 社交 & 2011\\
	\hline
	抖音 & Douyin.com & 社交 & 7441 \\
	\hline
	饿了么 & Ele.me & 居家生活 & 16053\\
	\hline
	国泰君安 & gtja.com & 金融理财 & 7734\\
	\hline
	QQ邮箱 & Tencent & 实用工具 & 4879\\
	\hline
	腾讯新闻 & Tencent & 新闻资讯 & 5679\\
	\hline
	支付宝 & Alipay.com & 金融理财 & 2301\\
	\hline
	阿里健康 & alihealth.cn & 居家生活 & 22904\\
	\hline
	安居客 & anjuke.com & 居家生活 & 2340\\
	\hline
	百词斩 & baicizhan.com & 实用工具 & 674\\
	\hline
	百合婚恋 &  baihe.com & 居家生活 & 2452\\
	\hline
	贝壳找房 & bj.ke.com & 居家生活 & 7520\\
	\hline
	当当阅读 & dangdang.com & 图书阅读 & 2588\\
	\hline
	钉钉 & dingtalk.com & 居家生活 & 3468\\
	\hline
	丁香 & dxy.cn & 居家生活 & 4654\\
	\hline
	豆瓣 & douban.com & 社交 & 3998\\
	\hline
	火山小视频 & huoshan.com & 影音视听 & 2924\\
	\hline
	Keep & www.gotokeep.com & 居家生活 & 9646\\
	\hline
	秒拍 & miaopai.com & 社交 & 340\\
	\hline
	中国南方航空 & China Southern Airlines & 旅行交通 & 7656\\
	\hline
	拼多多 & pinduoduo.com & 购物 & 2660\\
	\hline
	蜻蜓FM & 影音视听 & 社交 & 2312 \\
	\hline
	去哪儿 & qunar.com & 居家生活 & 3294\\
	\hline
	Soul & soulapp.cn & 聊天 & 4406\\
	\hline
	天涯 & tianya.cn & 聊天 & 1058\\
	\hline
	天眼查 & tianyancha.com & 实用工具 & 6146\\
	\hline
	同花顺 & 10jqka.com.cn & 金融理财 & 5928\\
	\hline
	王者荣耀 & Tencent & 游戏 & 2524\\
	\hline
	闲鱼 & 2.taobao.com & 购物 & 3226\\
	\hline
	小米运动 & huami.com & 居家生活 & 2364\\
	\hline
	新浪财经 & 金融理财.sina.com.cn & 金融理财 & 3608\\
	\hline
	央视新闻 & news.cctv.com & 社交 & 1534 \\
	\hline
	有道云笔记 & note.youdao.com & 实用工具 & 2396\\
	\hline
	掌上生活 & cmbchina.com & 金融理财 & 4838\\
	\hline
	直播吧 & zhibo8.cc & 实用工具 & 4368\\
	\hline
	中国国际航空 & 中国国际航空 & 旅行交通 & 3343\\
	\hline
	作业帮 & Zybang.com & 实用工具 & 4692\\
	\hline
	\hline
	\textbf{总计} & - & - & \textbf{236871}\\
	\hline
\end{longtable}

\section{小结}


% 湖南大学的《安卓手机应用流量分析恶意行为检测技术研究》


% \begin{figure}[!htbp]
% 	\centering
% 	\includegraphics[width=0.80\textwidth]{data_collect_algo1}
% 	\bicaption{深度优先搜索Android程序的执行路径}{}
% 	\label{fig:data_collect_algo_1}
% \end{figure}

% \subsection{移动终端应用与行为识别与技术研究——肖新光}

% \begin{figure}[!htbp]
% 	\centering
% 	\includegraphics[width=0.80\textwidth]{data_collect_algo_2}
% 	\bicaption{西电硕士论文}{}
% 	\label{fig:data_collect_algo_2}
% \end{figure}


% \subsection{自动化的标注}
% 一种思路是一次性仅仅标注一个或者非常有限的程序,这样可以借助一些特定的规则,比如端口号,SNI进行标注。或者借助认为的判断,但是对于应用程序很多的情况而言,这种方案不现实。所以在产生数据的时候更加倾向于一次性开启很少的程序,虚拟化网卡,在这个网卡上抓取应用产生的流量,这样可以避免其他程序,如操作系统带来的干扰。或者使用如iptable的技术来控制ip,当然这样做的前提还是局限在单次抓取的流量很少的情况。

% 对于混合流量,尝试使用L7-filter和GT工具来标注混合数据集。L7-filter的实现是基于特征的关键字匹配,采用正则表达式的方式来对关键字进行描述和匹配,这种方法对于协议的识别有着更高的效率和准确率。GT是一款开源的网络流分析工具,主要包括四个部分:(1)gt客户端进程:运行在每个被监控的网络节点上,从主机内核获取每一个应用的名称,并且记录每一个应用所存在的数据流信息(2)数据包捕获用来在网络的路由器处捕获所有被监控的主机的数据包(3)数据库服务器用来存储gt客户端上应用名和对应的数据流信息(4)数据集标记:将捕获的应用数据根据应用名和对应的数据流信息进行标记,并且根据具体的应用将数据流进行分区。
% \begin{figure}[!htbp]
% 	\centering
% 	\includegraphics[width=0.80\textwidth]{gt_structure}
% 	\bicaption{开源的gt工具}{}
% 	\label{fig:gt_structure}
% \end{figure}
% \url{http://netweb.ing.unibs.it/~ntw/实用工具/gt/}

% 开源的数据集:\href{http://tstat.tlc.polito.it/}{TCP STatistic and Analysis Tool},\href{http://mawi.wide.ad.jp/mawi/samplepoint-F/2018/201809021400.html}{pcap格式抓包数据}

