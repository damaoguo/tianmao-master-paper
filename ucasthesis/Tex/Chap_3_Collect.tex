\chapter{Android应用流量自动化/半自动化采集的研究}\label{chap:collect}
\section{Android应用安装包数据集}
\section{对于Android程序UI遍历的研究(脚本,自动运行)}
\section{数据清洗(对于背景流量的研究)}
\subsection{背景流量的生成(不包含应用test)}
\subsection{流量的过滤(在test中去除背景流量)}
\subsection{去除交集(剩下的流量将会互斥)}
\subsection{基于Android程序进程的清洗}
\section{小结}


这一章主要从流量抓取的角度分析,主要的目标是:标记每一帧对应的应用,同一个应用产生的流量到达的时间点需要被记录下来( 目的是提取相关的时间特征,分析出流量随时间的分布。将同一个应用不同时间产生的流量进行归类。我们基于传统的方法对数据进行标记,例如基于规则,基于SNI等。当然,这种标记方法很有缺陷,并不能保证能够将所有的流量进行准确的分类,否则我们就没有后序工作的必要了。且在思维上,我们应该应该走出悖论,在我们标记的数据集上,我们学习到的方法是不能够超越我们定义的规则的,在这些数据上我们并不是要追求100\%的准确程度(理论上也达不到),我们学习新的方法的动机也不是处理已经标记的数据,因为这是没有意义的事情,我们的目的是当新的流量到来的时候,可能基于我们标记数据的规则不能够处理(如这一批流量中并没有SNI信息,并没有端口信息),有或者新来的数据使用这些规则并不能处理好,但是我们学习到的新的方法却非常契合。我们在采集数据的时候,主要是要考虑数据产生的真实性,完整性,可信性。不能够脱离真正的网络环境采集数据,那么面临的困难时如何就混杂情况下的数据区分出来,毕竟各种方案都有缺陷,在采集标注数据的时候这种缺陷是应该被容忍的。

这一章主要从流量抓取的角度分析,主要的目标是:标记每一帧对应的应用,同一个应用产生的流量到达的时间点需要被记录下来( 目的是提取相关的时间特征,分析出流量随时间的分布。将同一个应用不同时间产生的流量进行归类。在网络流中,流被定义为两个终端之间具有相同的五元组的连续数据包,五元组包括:源端口,目的端口,源ip地址,目的ip地址和传输层协议,单向流中数据的五元组完全相同,双向流五元组中的源地址、源端口和目的地址、目的端口互换。对于网络流中的数据包进行识别是本文的研究目标,涉及到根据应用负载进行识别,这类数据即为TCP数据段中的应用数据部分。一般情况下,我们认为三次握手中的第一个SYN数据包的到达时间即为TCP流的开始时间,而TCP流的结束时间则认为是FIN数据包或者RST数据包到达的时间。在数据进行标注的时候,可以采用多元组来对数据进行标记,例如<源地址,源端口,目的地址,目的端口,协议,SNI>等进行标记。相关工作中有尝试:数据集主要捕获于某高校的教育网出口处以及内网节点,捕获之初的数据是混合数据集,为了获得纯净的数据集,需要利用现有的技术对其进行标注。

我们以流为单位进行研究和标记,离线的数据流结构如下:
\begin{figure}[!htbp]
	\centering
	\includegraphics[width=0.80\textwidth]{flow_offline}
	\bicaption{离线流数据}{}
	\label{fig:flow_offline}
\end{figure}






数据采集的流程如下:
\begin{figure}[!htbp]
	\centering
	\includegraphics[width=0.80\textwidth]{Data-collect.pdf}
	\bicaption{数据采集}{}
	\label{fig:data_collect}
\end{figure}



\section{问题?}
主要问题是传统的方法存在缺陷:基于端口的方法识别准确率越来越低,因为许多服务采用的是动态端口,另外端口可以被手动指定,进一步降低端口识别技术的精度。基于DPI(深度包)的方法只能够识别明文流量,一旦服务采用了加密的方式,就会隐去载荷部分的特征,不再可见。Moore等人在2005年提出使用机器学习的方法进行网络流量分类,并且提出了248中特征。


\section{Android应用}
Android Malware Genome Project
\begin{figure}[!htbp]
	\centering
	\includegraphics[width=0.80\textwidth]{softwares}
	\bicaption{Android软件,湖南大学}{}
	\label{fig:softwares}
\end{figure}



\section{抓包工具的选择}


\subsection{wireshark}
\subsection{fiddler}

\section{PC端HTTPS流量抓取}
\subsection{单一进程或者程序产生的流量采集}
\subsection{自动化的流量产生工具}
借助自动化的工具产生大量的流量,避免使用人工进行采集。

\subsubsection{Selenium}
Selenium的使用:\url{https://www.seleniumhq.org},这一款工具主要是用来自动运行浏览器。可以在多种平台多种浏览器下运行,还支持多种编程语言和测试工具。

\subsubsection{UIcrawler}


\subsubsection{AppCrawler}

\subsubsection{AndroGenerator}

湖南大学的《安卓手机应用流量分析恶意行为检测技术研究》
首先下载了10,000个应用,系统由三部分构成:执行组件,流量解析器,网络流量生成器。

基于广告服务器域名的方法提取广告库网络流量

\subsubsection{基于深度优先的路径识别算法}
首先从Android应用的Manifest文件中提取应用启动的时候显示的Activity,将这些作为可执行路径的起点。。。

\begin{figure}[!htbp]
	\centering
	\includegraphics[width=0.80\textwidth]{data_collect_algo1}
	\bicaption{深度优先搜索Android程序的执行路径}{}
	\label{fig:data_collect_algo_1}
\end{figure}

\subsection{移动终端应用与行为识别与技术研究——肖新光}
使用爬虫爬取应用商店的应用的apk文件,之后得到一些配置文件,之后使用命令来控制手机端的tcpdump运行抓取流量,将抓取的流量保存在手机的路径下。见图4.5《西电硕士论文》

\begin{figure}[!htbp]
	\centering
	\includegraphics[width=0.80\textwidth]{data_collect_algo_2}
	\bicaption{西电硕士论文}{}
	\label{fig:data_collect_algo_2}
\end{figure}


\subsection{自动化的标注}
一种思路是一次性仅仅标注一个或者非常有限的程序,这样可以借助一些特定的规则,比如端口号,SNI进行标注。或者借助认为的判断,但是对于应用程序很多的情况而言,这种方案不现实。所以在产生数据的时候更加倾向于一次性开启很少的程序,虚拟化网卡,在这个网卡上抓取应用产生的流量,这样可以避免其他程序,如操作系统带来的干扰。或者使用如iptable的技术来控制ip,当然这样做的前提还是局限在单次抓取的流量很少的情况。

对于混合流量,尝试使用L7-filter和GT工具来标注混合数据集。L7-filter的实现是基于特征的关键字匹配,采用正则表达式的方式来对关键字进行描述和匹配,这种方法对于协议的识别有着更高的效率和准确率。GT是一款开源的网络流分析工具,主要包括四个部分:(1)gt客户端进程:运行在每个被监控的网络节点上,从主机内核获取每一个应用的名称,并且记录每一个应用所存在的数据流信息(2)数据包捕获用来在网络的路由器处捕获所有被监控的主机的数据包(3)数据库服务器用来存储gt客户端上应用名和对应的数据流信息(4)数据集标记:将捕获的应用数据根据应用名和对应的数据流信息进行标记,并且根据具体的应用将数据流进行分区。
\begin{figure}[!htbp]
	\centering
	\includegraphics[width=0.80\textwidth]{gt_structure}
	\bicaption{开源的gt工具}{}
	\label{fig:gt_structure}
\end{figure}
\url{http://netweb.ing.unibs.it/~ntw/tools/gt/}

开源的数据集:\href{http://tstat.tlc.polito.it/}{TCP STatistic and Analysis Tool},\href{http://mawi.wide.ad.jp/mawi/samplepoint-F/2018/201809021400.html}{pcap格式抓包数据}

\section{Android端HTTPS流量抓取}
android端的流量抓取和pc端存在一定的差别。
\subsection{android手机应用的自动安装}
使用爬虫分类别爬取下载手机apk文件,然后进行安装。(在我们的实验中应该首先研究少量的app,在这个研究基础上做更加深入的研究)



\begin{longtable}{c|c|c|c}
    \bicaption{Android应用}{Android apps}\\
	\hline
	\textbf{应用名} & \textbf{开发商} & \textbf{应用类别} & \textbf{应用数量} \\
	\hline
	\endfirsthead
	\multicolumn{4}{c}%
        {\bfseries\small \tablename\ \thetable\ {续表。}}\\
	\hline
	\textbf{应用名} & \textbf{开发商} & \textbf{应用类别} & \textbf{应用数量} \\
	\hline
	\endhead
	\hline \multicolumn{4}{r}{\textit{续表见下页}}\\
	\endfoot
	\hline
	\endlastfoot
	Baidu Map & Baidu.com & Travel \& Local & 6777\\
	\hline
	Baidu Post Bar & Baidu.com & Social & 3234\\
	\hline
	Netease cloud music & Netease.com & Music \& Audio & 9888\\
	\hline
	iQIYI &  iQIYI & Music \& Audio & 2634\\
	\hline
	JingDong & JD & Shopping & 7956\\
	\hline
	Jinritoutiao & ByteDance & News \& Magazines & 5321\\
	\hline
	Meituan & Meituan.com& Lifestyle & 12469\\
	\hline
	QQ & Tencent & Communication & 1246\\
	\hline
	QQ music & Tencent & Music \& Audio & 1155\\
	\hline
	QQ reader & Tencent & Books \& Reference & 1563\\
	\hline
	Taobao & Taobao & Shopping & 3431\\
	\hline
	Weibo & Sina & Social & 3097\\
	\hline
	CTRIP & CTRIP & Lifestyle & 2141\\
	\hline
	Zhihu & Zhihu.com & Social & 2011\\
	\hline
	Tik Tok & Douyin.com & Social & 7441 \\
	\hline
	Ele.me & Ele.me & Lifestyle & 16053\\
	\hline
	gtja & gtja.com & Finance & 7734\\
	\hline
	QQ mail & Tencent & Tools & 4879\\
	\hline
	Tencent News & Tencent & News \& Magazines & 5679\\
	\hline
	Alipay & Alipay.com & Finance & 2301\\
	\hline
	ALiJianKang & alihealth.cn & Lifestyle & 22904\\
	\hline
	AnJuKe & anjuke.com & Lifestyle & 2340\\
	\hline
	BaiCiZhan & baicizhan.com & Tools & 674\\
	\hline
	BaiHeHunLian &  baihe.com & Lifestyle & 2452\\
	\hline
	BeiKeZhangFang & bj.ke.com & Lifestyle & 7520\\
	\hline
	DangDangYueDu & dangdang.com & Books \& Reference & 2588\\
	\hline
	DingDing & dingtalk.com & Lifestyle & 3468\\
	\hline
	DingXiang & dxy.cn & Lifestyle & 4654\\
	\hline
	DouBan & douban.com & Social & 3998\\
	\hline
	VigoVideo & huoshan.com & Music \& Audio & 2924\\
	\hline
	Keep & www.gotokeep.com & Lifestyle & 9646\\
	\hline
	MiaoPai & miaopai.com & Social & 340\\
	\hline
	csair App & China Southern Airlines & Travel \& Local & 7656\\
	\hline
	PinDuoDuo & pinduoduo.com & Shopping & 2660\\
	\hline
	QingTing FM & Music \& Audio & Social & 2312 \\
	\hline
	QuNaEr & qunar.com & Lifestyle & 3294\\
	\hline
	Soul & soulapp.cn & Communication & 4406\\
	\hline
	TianYa & tianya.cn & Communication & 1058\\
	\hline
	TianYanCha & tianyancha.com & Tools & 6146\\
	\hline
	TongHuaShun & 10jqka.com.cn & Finance & 5928\\
	\hline
	Arena of Valor & Tencent & Game & 2524\\
	\hline
	XianYu & 2.taobao.com & Shopping & 3226\\
	\hline
	XiaoMiYunDong & huami.com & Lifestyle & 2364\\
	\hline
	XinLangCaiJing & finance.sina.com.cn & Finance & 3608\\
	\hline
	YangShiXinWen & news.cctv.com & Social & 1534 \\
	\hline
	YouDaoYunBiJi & note.youdao.com & Tools & 2396\\
	\hline
	ZhangShangShengHuo & cmbchina.com & Finance & 4838\\
	\hline
	ZhiBoBa & zhibo8.cc & Tools & 4368\\
	\hline
	airchina App & Air China & News \& Travel \& Local & 3343\\
	\hline
	ZuoYeBang & Zybang.com/ & Tools & 4692\\
	\hline
	\hline
	\textbf{Total} & - & - & \textbf{107010}\\
	\hline
\end{longtable}



\subsection{monkey自动运行程序}


\subsection{VPN的HTTPS流量抓取}


\subsection{使用anyproxy拦截https请求}
利用中间人攻击的方法,前提是需要客户端提前信任anyproxy产生的证书。参考 \url{anyproxy.io.cn}当代理服务器收到https请求时,AnyProxy可以替换证书,对请求做明文解析。调用规则模块beforeDealHttpsRequest方法,如果返回true,会明文解析这个请求,其他请求不处理。被明文解析后的https请求,处理流程同http一致。未明文解析请求不会再进入规则模块做处理。

\section{抓取的文件导出为XML文件}
将文件保存为键值对的形式,便于使用相关的工具进行处理,提取信息。主要的字段包括:

<packet>
	<proto name="" pos="" showname="" size=>
		<field name="" pos="" show="" showname="" value= size= >
			<field name="" pos="" ...>
			...
		</field>
		...
	</proto>
	...
</packet>

各个字段相互嵌套,没一条记录了的第一个字段name是我们解析后用来分析数据的主键。一个packet包含了多个proto,一个proto中又包含了多个字段filed ,保存的时候选择保存为PDML XML将会保存完整的信息。使用XML相关工具,提取XML的结构和字段信息。主要是为了弄清楚没一个proto可能包含的字段,从整体上对https流量进行分析。主要的一些字段包括:
SNI: 
\begin{figure}[!htbp]
	\centering
	\includegraphics[width=0.80\textwidth]{SNI_extensions_server_name}
	\bicaption{client hello中的SNI}{}
	\label{fig:datagram}
\end{figure}

时间戳:
\begin{figure}[!htbp]
	\centering
	\includegraphics[width=0.80\textwidth]{time_stamp}
	\bicaption{时间戳}{}
	\label{fig:datagram}
\end{figure}

