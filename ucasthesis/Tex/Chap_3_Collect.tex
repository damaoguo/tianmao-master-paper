\chapter{自动化移动应用HTTPS流量标记方法}\label{chap:collect}
在基于HTTPS流量的移动应用识别技术研究中,需要大量的、有标记的数据的支持,所谓的标记就是产生流量的应用的应用名,而不是使用的协议类别等标签。然而,当前所公布的开源数据较少,数据量有限,流量数据标记较为粗糙,不满足本文的研究工作的需要。本章详细地描述了移动应用HTTPS流量数据地采集和处理方法,构建了一个覆盖11种应用类别、涉及50个移动应用、共计约20多万条HTTPS加密网络数据流的有标记数据集。

\section{引言}


本章首先本文调研了当前已经公开的流量数据集,总结这些公开数据集的来源、规模、协议、用途以及获取方式。现有的公开并被多项研究工作使用的数据集如下:
\begin{itemize}
    \item \citep{draper2016characterization}发布了VPN-nonVPN dataset (ISCXVPN2016)数据集\footnote{https://www.unb.ca/cic/datasets/vpn.html},其中包括7种常规加密流量和7种协议封装流量。该项工作捕获的流量包含常规会话和通过VPN的会话,共有14种流量类别,如VOIP,VPN-VOIP,P2P,VPN-P2P等,数据集内容包括见表~\ref{tab:sample}。
    \begin{table}[!htbp]
        \bicaption{ISCXVPN2016 数据集。}{ISCXVPN2016 Dataset.}
        \label{tab:sample}
        \centering
        \footnotesize% fontsize
        \setlength{\tabcolsep}{4pt}% column separation
        \renewcommand{\arraystretch}{1.2}%row space 
        \begin{tabular}{lc}
            \hline
            类型 & 内容 \\
            \hline
            Traffic & Content \\
            Web Browsing & Firefox and Chrome \\
            Email & SMPTS, POP3S and IMAPS \\
            Chat & ICQ, AIM, Skype, Facebook and Hangouts \\
            Streaming & Vimeo and Youtube \\
            File Transfer & Skype, FTPS and SFTP using Filezilla and an external service \\
            VoIP & Facebook, Skype and Hangouts voice calls (1h duration) \\
            P2P & uTorrent and Transmission (Bittorrent)\\
            \hline
        \end{tabular}
    \end{table}
    
    \item Intrusion detection evaluation dataset (ISCXIDS2012) \footnote{https://www.unb.ca/cic/datasets/ids.html}:该数据集包括标记的网络跟踪、完整数据包有效负载(pcap格式)以及相关的配置文件,供研究人员公开使用。如表\ref{tab:ISCXIDS2012}所示,该数据集由7天的网络活动(正常和恶意)组成。该数据集合主要针对恶意流量的分析,包含了HTTP,SSH等协议诸如DDOS,Brute Force等恶意行为,因此该数据集在本文研究的任务重合度有限。
     \begin{table}[!htbp]
        \bicaption{ISCXIDS2012数据集。}{ISCXIDS2012 Dataset.}
        \label{tab:ISCXIDS2012}
        \centering
        \footnotesize% fontsize
        \setlength{\tabcolsep}{4pt}% column separation
        \renewcommand{\arraystretch}{1.2}%row space 
        \begin{tabular}{lcc}
        \hline
        \textbf{日期} &  \textbf{描述} & \textbf{数据规模(GB)}\\
        \hline
        11/6/2010 & Normal Activity. No malicious activity & 16.1\\
        12/6/2010 & Normal Activity. No malicious activity & 4.22\\
        13/6/2010 & Infiltrating the network from inside + Normal Activity & 3.95\\
        14/6/2010 & HTTP Denial of Service + Normal Activity & 6.85\\
        15/6/2010 & Distributed Denial of Service using an IRC Botnet & 23.4\\
        16/6/2010 & Normal Activity. No malicious activity & 17.6\\
        17/6/2010 & Brute Force SSH + Normal Activity & 12.3\\
        \hline
        \end{tabular}
    \end{table}
    
    
    \item \citep{bujlow2015independent}提出数据集Independent Comparison of Popular DPI Tools for Traffic Classification" dataset\footnote{https://cba.upc.edu/monitoring/traffic-classification},这个数据集包含767,690个流,这些流占53.31GB的纯数据包数据。存在759,720个数据流的应用程序名称(占所有流的98.96%),占数据量的51.93GB(97.41%)。数据集由一个pcap跟踪和一个INFO文件组成,其中INFO文件中的每一行对应于pcap跟踪中的流,并描述如下:
    \begin{lstlisting}
     flow_id + "#" + start_time + "#" + end_time + "#" + local_ip + "#" + remote_ip + "#" + local_port + "#" + remote_port + "#" + transport_protocol + "#" + operating_system + "#" + process_name + "#" + HTTP Url + "#" + HTTP Referer + "#" + HTTP Content-type +"#"
    \end{lstlisting}
    本数据集数据规模较大,但是数据是已经预处理,对于研究新的识别方法的帮助有限。
    
    \item DARPA99 traces\footnote{https://www.ll.mit.edu/r-d/datasets}:该ARPA99跟踪是来自MIT Lincoln实验室1999年的模拟网络的数据包。这些数据包中提供了所有网络数据,其标签是不同的攻击行为,因此该数据集不符合本文的研究场景。
    
    \item MAWILab\footnote{http://mawi.wide.ad.jp/mawi/}:\citep{mawilab}发布了MAWILab数据集,可帮助研究人员评估流量异常检测方法。它由一组用于定位MAWI存档中异常流量的标签组成。使用基于图形的高级方法获得标签,该方法比较并组合了不同且独立的异常检测器。数据集每天更新,以包括来自即将到来的应用程序和异常的新流量,用于评估异常流量检测方法的,因此不符合本文的研究场景。
    
    \item NLANR AMP Data\footnote{https://labs.ripe.net/datarepository/data-sets/nlanr-amp-data}:此数据是NLANR研究小组收集的一组活动测量(ping /traceroute)。 数据由多达130个有利点的网格中的测量组成,并且这些测量在1998年至2006年之间进行。该数据可用于Internet的纵向研究。RIPE数据存储库中的可用数据是NLANR网站上可用的原始数据的重新制作版本。该数据集包括的是\emph{ping/traceroute}和相关的流量,不符合本文的研究场景所需。
    
    \item NIMS\footnote{https://projects.cs.dal.ca/projectx/Download.html}:在研究工作中\citep{alshammari2011can,alshammari2008investigating,alshammari2007flow}提出了NIMS数据集,该数据的标签为:\emph{TELNET, FTP, HTTP, DNS, lime, localForwarding, remoteForwarding, scp, sftp, x11, shell},是基于协议层的分类,且单条样本已经被预处理为22个字段的数值,主要为均值、方差、最大值、最小值等统计特征。该数据集涉及了多种加密协议的样本,在识别粒度上要比本文的识别任务粗糙,不符合本文场景。
    
    \item WITS: Waikato Internet Traffic Storage\footnote{https://wand.net.nz/wits/}:该数据集目前有33个不同的集合构成,数据规模较为庞大,且数据来源较为丰富,但是该数据集采集的是混合流量,样本缺少精细化的标记,因此难于用于训练精细化的流量识别方法,无法满足本文中的精细化识别场景的需求。
    
    
\end{itemize}


本课题的目的在于通过对大量的应用的HTTPS流量数据进行分析,得到一种可靠准确的分析方法,经过分析,以上提到的公开数据集在通信协议、数据规模、数据标签、数据格式或者使用场景等不满足本文针对HTTPS流量进行精细化识别的要求。为此需要构建新的数据集来支撑本文的研究工作。

\section{方法阐述}
在本文章中介绍一种自动化移动应用HTTPS流量标记方法。数据采集的流程如图所示,如图\ref{fig:data-collect}所示,该系统主要由APP爬虫、应用调度、App运行环境、事件注入和守护进程五个模块构成,具体介绍如下:
\begin{figure}[!htbp]
	\centering
	\includegraphics[width=0.80\textwidth]{Data-collect.pdf}
	\bicaption{数据采集}{Data collect}
	\label{fig:data-collect}
\end{figure}
\begin{itemize}
    \item APP爬虫:目前市场上的移动应用数量规模巨大,手动下载安装程序是不切实际的。为了便捷地从应用市场下载应用,首先设计了爬虫工具,为爬虫提供下载连接等信息,爬虫工具提取下载地址,从应用市场爬取移动应用的安装包。将爬取的移动应用安装包存储,以用于异步分发。
    \item 应用调度:使用调度模块选取安装文件进行选取、提取信息(获得应用名,用作流量标签)、分发,在应用的运行环境完成安装。为了给抓取的流量添加唯一的标签,同一个流量采集过程中,每一个设备只分发安装一个应用程序。
    \item APP运行环境:运行环境是由移动真机设备和模拟器组成的,其提供了流量抓取目标程序的运行环境,为了避免设备本身的影响,同一个应用程序在不同的设备上运行,并通过限制设备中其他应用的网络权限来确保没有应用在后台运行。
    \item 事件注入:通过一些列操作事件驱动移动应用自动运行,借助模糊测试工具monkeyrunner\footnote{https://developer. android. com/studio/test/monkeyrunner/index.html}产生事件模拟真实的人工操作场景。移动应用在运行的过程中将产生大量的网络流量。
    \item 守护进程:该模块中的流量捕获模块(核心为Wireshark \footnote{https://www.wireshark.org/}、TcpDump\footnote{http://www.tcpdump.org}等)用于抓取流量。另外在整个运行的过程中,真机和模拟器中的模糊测试模糊测试事件不可避免产生诸多错误,表现为移动应用闪退、不响应等问题。守护进程除了捕获流量外,还承担错误判断、错误中止、清除脏数据、重新启动移动应用和捕获过程等重要功能。
    
\end{itemize}


\section{数据集描述}

本文以流为单位进行研究和标记,在采集流量的时候,保存的同一个pcap文件是由多条流构成的。在流量被采集后,通过五元组信息---\textit{[源地址,源端口,目的地址,目的端口,传输协议]}将pcap文件进行切分,pcap文件中的数据包重组后形成新的pcap文件,被切分后,每一个pcap文件都是一条样本,本文中使用的抽取方法介绍如下。

(1)基于splitcap抽取流:SplitCap\footnote{https://www.netresec.com/?page=SplitCap}是一个免费工具,旨在根据诸如IP地址,5元组或MAC地址等标准将捕获文件(PCAP文件)拆分为较小的文件。 可用于拆分/分组的标准是:
\begin{itemize}
    \item BSSID:根据WLAN BSSID分组的数据包
    \item 流:每个5元组的单向流量(传输协议,IP地址和端口号)分组在一起。
    \item 主机:将流量按IP地址(源和目标)分组到一个文件。 大多数数据包将以两个文件结尾。
    \item 主机对:基于IP对通信进行分组的流量。
    \item MAC地址:将流量按每个MAC地址分组到一个文件。 大多数数据包将以两个文件结尾。
    \item 会话:每个会话的数据包(双向流)被分组在一起。
    \item 时间:根据时间拆分。
    \item 数据包计数:根据数据包计数拆分。
\end{itemize}
本文中选择按照流的标准来切分流,具体的操作代码如下:
\begin{lstlisting}[language=sh]
foreach($f in gci 1_Pcap *.pcap)
{
    SplitCap -p 100000 -b 100000 -r $f.FullName -o 2_Session\AllLayers\$($f.BaseName)-ALL
    SplitCap -p 100000 -b 100000 -r $f.FullName -s flow -o 2_Session\AllLayers\$($f.BaseName)-ALL
    gci 2_Session\AllLayers\$($f.BaseName)-ALL | ?{$_.Length -eq 0} | del
}
\end{lstlisting}

(2)基于Scapy抽取流:Scapy是一个Python程序,使用户能够发送,嗅探,剖析和伪造网络数据包。 此功能允许构建可以探测,扫描或攻击网络的工具。Scapy是功能强大的交互式数据包处理程序。 它能够伪造或解码各种协议的数据包,在线发送它们,捕获它们,匹配请求和答复等等。 Scapy可以轻松处理大多数经典任务,例如扫描,跟踪路由,探测,单元测试,攻击或网络发现。它可以替代\textit{hping,arpspoof,arp-sk,arping,p0f}甚至\textit{Nmap,tcpdump和tshark}的某些部分。

本文中利用Scapy按照五元组信息重组pcap文件,提取流,并将提取的流保存为pcap文件格式,这里附上提取代码\footnote{http://www.damaoguo.site/2020/03/06/ExtractFlowWithPython/}。

原始的流量数据经过处理以后,一个原始的pacp文件将会按照流拆分为新的pcap小文件,文件存储的目录结构如图\ref{fig:file-tree}所示。
\begin{figure}[!htbp]
	\centering
	\includegraphics[width=0.45\textwidth]{File-tree.pdf}
	\bicaption{流数据文件存储目录}{Traffic Flow File Tree}
	\label{fig:file-tree}
\end{figure}

下表展示了所构建的数据集的信息,共采集了旅行交通、社交、影音视听、时尚购物新闻资讯、居家生活、聊天、图书阅读、金融理财、实用工具、游戏共计11种类型,50个应用的流量,使用sliptcap切分后得到20万多万条流量样本。

\begin{longtable}{cccc}
    \bicaption{移动应用HTTPS流量数据集}{mobile apps HTTPS Traffic Dataset}\\
	\hline
	\textbf{应用名称} & \textbf{开发商} & \textbf{应用类别} & \textbf{流数量} \\
	\hline
	\endfirsthead
	\multicolumn{4}{c}%
        {\bfseries\small \tablename\ \thetable\ {续表。}}\\
	\hline
	\textbf{应用名} & \textbf{开发商} & \textbf{应用类别} & \textbf{应用数量} \\
	\hline
	\endhead
	\hline \multicolumn{4}{r}{\textit{续表见下页}}\\
	\endfoot
	\hline
	\endlastfoot
	百度地图 & Baidu.com & 旅行交通 & 6777\\
	百度贴吧 & Baidu.com & 社交 & 3234\\
	网易云音乐 & Netease.com & 影音视听 & 9888\\
	爱奇艺 &  iQIYI & 影音视听 & 2634\\
	京东 & JD & 购物 & 7956\\
	今日头条 & ByteDance & 新闻资讯 & 5321\\
	美团 & Meituan.com& 居家生活 & 12469\\
	QQ & Tencent & 聊天 & 1246\\
	QQ音乐 & Tencent & 影音视听 & 1155\\
	QQ阅读 & Tencent & 图书阅读 & 1563\\
	淘宝 & Taobao & 购物 & 3431\\
	微博 & Sina & 社交 & 3097\\
	携程 & CTRIP & 居家生活 & 2141\\
	知乎 & Zhihu.com & 社交 & 2011\\
	抖音 & Douyin.com & 社交 & 7441 \\
	饿了么 & Ele.me & 居家生活 & 16053\\
	国泰君安 & gtja.com & 金融理财 & 7734\\
	QQ邮箱 & Tencent & 实用工具 & 4879\\
	腾讯新闻 & Tencent & 新闻资讯 & 5679\\
	支付宝 & Alipay.com & 金融理财 & 2301\\
	阿里健康 & alihealth.cn & 居家生活 & 22904\\
	安居客 & anjuke.com & 居家生活 & 2340\\
	百词斩 & baicizhan.com & 实用工具 & 674\\
	百合婚恋 &  baihe.com & 居家生活 & 2452\\
	贝壳找房 & bj.ke.com & 居家生活 & 7520\\
	当当阅读 & dangdang.com & 图书阅读 & 2588\\
	钉钉 & dingtalk.com & 居家生活 & 3468\\
	丁香 & dxy.cn & 居家生活 & 4654\\
	豆瓣 & douban.com & 社交 & 3998\\
	火山小视频 & huoshan.com & 影音视听 & 2924\\
	Keep & www.gotokeep.com & 居家生活 & 9646\\
	秒拍 & miaopai.com & 社交 & 340\\
	中国南方航空 & China Southern Airlines & 旅行交通 & 7656\\
	拼多多 & pinduoduo.com & 购物 & 2660\\
	蜻蜓FM & 影音视听 & 社交 & 2312 \\
	去哪儿 & qunar.com & 居家生活 & 3294\\
	Soul & soulapp.cn & 聊天 & 4406\\
	天涯 & tianya.cn & 聊天 & 1058\\
	天眼查 & tianyancha.com & 实用工具 & 6146\\
	同花顺 & 10jqka.com.cn & 金融理财 & 5928\\
	王者荣耀 & Tencent & 游戏 & 2524\\
	闲鱼 & 2.taobao.com & 购物 & 3226\\
	小米运动 & huami.com & 居家生活 & 2364\\
	新浪财经 & 金融理财.sina.com.cn & 金融理财 & 3608\\
	央视新闻 & news.cctv.com & 社交 & 1534 \\
	有道云笔记 & note.youdao.com & 实用工具 & 2396\\
	掌上生活 & cmbchina.com & 金融理财 & 4838\\
	直播吧 & zhibo8.cc & 实用工具 & 4368\\
	中国国际航空 & 中国国际航空 & 旅行交通 & 3343\\
	作业帮 & Zybang.com & 实用工具 & 4692\\
	\hline
	\textbf{总计} & - & - & \textbf{236871}\\
	\hline
\end{longtable}

\section{本章小结}
本章提出了一个自动化的移动应用HTTPS流量采集标记方法,并实现了相关的系统,关键的技术分别是应用的自动执行、大规模流量生成和流量标记问题。通过本章的研究,为本文的研究工作提供了可靠的流量数据。