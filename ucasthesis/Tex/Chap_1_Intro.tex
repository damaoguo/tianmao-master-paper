\chapter{绪论}\label{chap:introduction}

\section{研究背景及研究的目的和意义}
移动互联网的迅速发展导致移动应用程序数量急剧增加。截至2018年12月,在中国市场受监控的移动应用程序(Apps)数量为449万\citet{cnnic2019}。随着移动互联网的快速发展和普及,越来越多的人开始使用互联网。HTTPS(Hyper Text Transfer Protocol Secure)的使用越来越广泛,例如,处于服务安全的考虑,谷歌正在投资并努力确保其网站和服务默认提供现代HTTPS。它的目标是在其产品和服务上实现100%的加密\citet{google2018google}。

因此,\underline{e}crypted \underline{m}obile \underline {a}pplication\ underline {i}dentification(简称EMAI)\emph{将加密的SSL/TLS流与特定的移动应用程序关联}作为许多任务场景的基本技术,这是一项至关重要的挑战,例如流量监控,QoS保证,网络异常检测等。综上所述,随着移动互联网的快速发展,针对移动互联网的流量的分析工作越来越重要。使用深度学习技术,可以充分利用流量数据,研究出一种快速、准确的识别技术,在网络管理上具有重要的意义。
本文主要针对各种应用HTTPS流量进行分析,HTTPS应用流量分分析具有非常重要的意义:

我们的目标之对加密的网络流量进行精细化分类:
\begin{figure}[!htbp]
	\centering
	\includegraphics[width=0.80\textwidth]{detail_classification}
	\bicaption{网络流量精细化分类}{}
	\label{fig:gt_structure}
\end{figure}

\section{国内外相关技术及其研究现状}


\subsection{基于协议本身的HTTPS流量识别}
基于DPI(深度包)的方法只能够识别明文流量,一旦服务采用了加密的方式,就会隐去载荷部分的特征,不再可见。
\subsubsection{基于SNI的HTTPS流量应用识别}

基于SNI的HTTPS流量应用识别技术是一种广泛使用的实用技术,用于识别HTTPS流量,一种基于SNI的监视,它使用服务器名称指示(SNI)(SSL / TLS握手的字段)来识别HTTPS连接中访问的网站。 SNI是TLS ClientHello消息中的明文字符串值,它提供了一种方便的方法来了解新HTTPS连接所访问的服务。 基于SNI的监视已集成到许多防火墙解决方案(Sphire wall\footnote{http://www.sphirewall.net},Untangle NG\footnote{https://www.untangle.com},IPFire3\footnote{http://www.ipfire.org}等)中。下图是将百度的HTTPS流量使用wireshark转为xml文件后,可见SNI为”www.baidu.com“。

\begin{figure}[!htbp]
	\centering
	\includegraphics[width=0.80\textwidth]{SNI-extensions-server-name.jpg}
	\bicaption{SNI}{SNI}
	\label{fig:SNI-extensions-server-name}
\end{figure}

\citet{Shbair2016Improving}实现基于SNI给流量添加标签。这篇文章主要的思路是:利用域名数据集( 两个数据集\footnote{https://scans.io/series/443-https-tls-alexa_top1mil},\footnote{https://cs.uwaterloo.ca/t55wang/wf.html},使用爬虫工具进行访问,获取到在建立握手的Client hello过程中返回的SNI,为了识别SNI是否是真实的,又结合了DNS解析SNI的结果和真实的ip进行比较,方法是:通过ip请求域名服务器,得到该ip所对应的域名,将得到的域名和SNI进行比较,判断该SNI是否正确。基于SNI的技术不能够处理SNI被移除的流,也不能够处理SNI被造假的情况。

\subsubsection{基于message type的HTTPS流量应用应用识别}
\citep{shen2017classification}利用TLS嵌入的消息类型作为状态,并在数据包级别上建立了隐马尔可夫模型。它主要利用HTTPS的消息类型(它是通信过程中协议设计的功能)作为状态信息,并扩展了应用程序的指纹。并验证数据包的消息类型功能可用于流量识别。

\subsection{基于机器学习的HTTPS流量识别}
近些年来,使用机器学习和深度学习在流量识别研究领域成为一种趋势。在过去的十年中,许多统计和机器学习算法已应用于流量分类问题。针对HTTP流量,\citet{Moore2005Internet}在2005年提出使用机器学习的方法进行网络流量分类,并且提出了248中特征。这248中特征具有普适性,能够迁移到处理各种协议的分析工作。这些特征主要包括:packet数量,packet头部大小,TCP分片大小,packet到达的时间等时间和大小上的特征,并计算均值方差等统计量,计算熵以及傅里叶变换等。同时,这项工作还选取特征并证明其在分类任务中的重要性。HTTPS流量识别的现有技术通常采用基于统计功能的机器学习方法\citep{Muehlstein2016Analyzing, Kohout2017Network}。\citep{Wright2006On}提出了一种基于大小和时间的特征提取方法。 此方法完全独立于内容。 Fahad等人\citep{Fahad2013Toward}提出了一种选择交通特征的方法,其主要目的是获得最重要的交通特征。 Jonathan等人\citep{Muehlstein2016Analyzing}展示的方法专注于加密网络流量的统计分析,以识别用户的操作系统,浏览器和应用程序。 它们提供了利用浏览器的突发行为和SSL(安全套接字层)行为的新功能。 这些基于统计特征的方法依赖于复杂的特征构造和提取。

基于机器学习的HTTPS流量精细化识别的主要缺点是功能设计和选择的复杂性,当网络环境变化时,这将导致通用性较差。

\subsection{基于深度学习的HTTPS流量识别}
Yu Zhang等\citep{Chen2017Automatic}提出了一种基于HTTP流量识别自动移动应用程序的方法。 他们使用卷积神经网络自动提取特征,并为每种应用建立了二进制分类模型。 但是,应用程序市场上有大量应用程序,所有应用程序的模型训练和识别的时间复杂度将非常高。 \citep{Wang2017End}等提出了一种端到端流量分类方法,该方法提取忽略任何其他信息的流或会话的前784个字节,然后使用一维卷积进行处理,




\section{论文主要研究内容}
针对当前HTTPS应用识别存在的依赖特征工程高、特征建模能力弱、识别粒度粗、迁移能力弱等问题,本课题旨在研究精细化的HTTPS网络应用特征提取与识别方法。
\subsection{自动化流量数据标记方法}
主要解决由于标记粗糙导致的识别粒度粗的问题。

\subsection{基于多视图序列特征的应用识别方法}
解决特征定义选择困难,依赖特征工程高,特征建模能力弱的问题。

\subsection{基于增量学习的新应用识别方法}
解决新增应用识别问题。

\section{论文组织架构}
本文的组织结构如下:

第一章:绪论。在本章中,首先介绍了移动应用数量快速增长的趋势。随后介绍了针对针对移动应用流量分类问题的研究现状、基于机器学习的识别方法、基于深度学习的识别方法,分别从基于协议本身的识别方法。最后概述本文主要的研究内容和本文的组织结构。

第二章:相关研究概述。本课题主要研究的是HTTPS流量的数据,本章首先对HTTPS协议进行了研究,重点关注了HTTPS握手过程过程以及协议本身定义的参数。接着我们介绍在本研究工作中使用到的机器学习、深度学习、增量学习方法,对不同的模型进行了阐述并调研了相关的研究工作。

第三章:数据集和数据预处理。首先调研了现在存在的公开数据集,对不同的数据集进行了分析,并分析在本课题中公开的数据集不能够使用的原因。接着我们详细介绍了精细化的流量数据采集标定方法,并构建了针对Android应用的流量数据。

第四章:基于HTTPS流量的HTTPS流量应用多视图特征识别方法的研究。从三个不同的视图方向进行研究,使用深度学习的方法提取抽象的特征,训练识别模型,针对20种应用进行实验,验证我们提出的AIBMF模型的有效性。

第五章:基于增量学习的HTTPS新应用识别。在真实的场景中,移动应用的数目在不断出现和更新,然而深度学习的方法在处理类别增加的问题的时候,会面临“灾难遗忘”的问题,导致模型的识别能力快速下降。本章提出了InceAIBMF模型用于针对新应用持续增加的场景。并重点关注该模型在训练时候的性能消耗,以及时间消耗。

第六章:总结和展望。本章总结了本文的主要成果,并对本文所提出的应用及其进行了扩展。
