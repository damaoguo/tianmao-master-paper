\chapter{绪论}\label{chap:introduction}

\section{研究背景及研究的目的和意义}
移动互联网的迅速发展导致移动应用程序数量急剧增加。截至2018年12月,在中国市场受监控的移动应用程序(Apps)数量为449万\citet{cnnic2019}。随着移动互联网的快速发展和普及,越来越多的人开始使用互联网。HTTPS(Hyper Text Transfer Protocol Secure)的使用越来越广泛,例如,处于服务安全的考虑,谷歌正在投资并努力确保其网站和服务默认提供现代HTTPS。 它的目标是在其产品和服务上实现100%的加密\citet{google2018google}。

因此,\underline{e}crypted \underline{m}obile \underline {a}pplication\ underline {i}dentification(简称EMAI)\emph{将加密的SSL/TLS流与特定的移动应用程序关联}作为许多任务场景的基本技术,这是一项至关重要的挑战,例如流量监控,QoS保证,网络异常检测等。综上所述,随着移动互联网的快速发展,针对移动互联网的流量的分析工作越来越重要。使用深度学习技术,可以充分利用流量数据,研究出一种快速、准确的识别技术,在网络管理上具有重要的意义。

\section{国内外相关技术及其研究现状}
近些年来,使用机器学习和深度学习在流量识别研究领域成为一种趋势。在过去的十年中,许多统计和机器学习算法已应用于交通分类问题。但是,作者使用不同的方法学数据集来评估其方法,因此结果无法直接比较。 大多数方法使用监督或半监督机器学习算法对流进行分类,甚至确定流的应用协议。大多数方法都针对加密协议,例如安全外壳(SSH),安全套接字层(SSL)/传输层安全性(TLS)和加密的BitTorrent。

\subsection{基于协议本身的HTTPS流量识别}
基于DPI(深度包)的方法只能够识别明文流量,一旦服务采用了加密的方式,就会隐去载荷部分的特征,不再可见。
\subsubsection{基于SNI的HTTPS流量应用识别}

基于SNI的HTTPS流量应用识别技术是一种广泛使用的实用技术,用于识别HTTPS流量,一种基于SNI的监视,它使用服务器名称指示(SNI)(SSL / TLS握手的字段)来识别HTTPS连接中访问的网站。 SNI是TLS ClientHello消息中的明文字符串值,它提供了一种方便的方法来了解新HTTPS连接所访问的服务。 基于SNI的监视已集成到许多防火墙解决方案(Sphirewall\footnote{http://www.sphirewall.net},Untangle NG\footnote{https://www.untangle.com},IPFire3\footnote{http://www.ipfire.org}等)中。




\subsubsection{基于message type的HTTPS流量应用应用识别}


\subsection{基于机器学习的HTTPS流量识别}
针对HTTP流量,\citet{Moore2005Internet}在2005年提出使用机器学习的方法进行网络流量分类,并且提出了248中特征。这248中特征具有普适性,能够迁移到处理各种协议的分析工作。这些特征主要包括:packet数量,packet头部大小,TCP分片大小,packet到达的时间等时间和大小上的特征,并计算均值方差等统计量,计算熵以及傅里叶变换等。同时,这项工作还选取特征并证明其在分类任务中的重要性。

\subsection{基于深度学习的HTTPS流量识别}


\subsection{细粒度的应用识别技术}


\subsection{传统的统计特征应用识别}

\citet{Shbair2016Improving}实现基于SNI给流量添加标签。这篇文章主要的思路是:利用域名数据集( 两个数据集:\href{https://scans.io/series/443-https-tls-alexa_top1mil}{数据集1},\href{https://cs.uwaterloo.ca/ t55wang/wf.html}{数据集2}),使用爬虫工具进行访问,获取到在建立握手的Client hello过程中返回的SNI( 其实是一个扩展extension,名称为server name),为了识别SNI是否是真是的又结合了DNS解析SNI的结果和真实的ip进行比较。
\begin{figure}[!htbp]
	\centering
	\includegraphics[width=0.80\textwidth]{SNI_vertification}
	\bicaption{SNI结合DNS验证}{}
	\label{fig:SNI_vertification}
\end{figure}
\subsection{存在的问题}




\section{论文主要研究内容}

本文主要针对各种应用HTTPS流量进行分析,HTTPS应用流量分分析具有非常重要的意义:

我们的目标之对加密的网络流量进行精细化分类:
\begin{figure}[!htbp]
	\centering
	\includegraphics[width=0.80\textwidth]{detail_classification}
	\bicaption{网络流量精细化分类}{}
	\label{fig:gt_structure}
\end{figure}

\section{论文组织架构}
本文的组织结构如下:

第一章:绪论。在本章中,首先介绍了移动应用数量快速增长的趋势。随后介绍了针对针对移动应用流量分类问题的研究现状。最后概述本文主要的研究内容和本文的组织结构。

第二章:相关研究概述。

第三章:数据集和数据预处理

第四章:基于HTTPS流量的HTTPS流量应用多视图特征识别方法的研究。

第五章:基于增量学习的HTTPS新应用识别。

第六章:总结和展望。本章总结了本文的主要成果,并对本文所提出的应用及其进行了扩展。
