\chapter{绪论}\label{chap:introduction}

\section{研究背景及研究的目的和意义}
基于DPI(深度包)的方法只能够识别明文流量,一旦服务采用了加密的方式,就会隐去载荷部分的特征,不再可见。Moore等人在2005年提出使用机器学习的方法进行网络流量分类,并且提出了248中特征。

\section{国内外相关技术及其研究现状}
近些年来,使用机器学习和深度学习在流量识别研究领域成为一种趋势。在过去的十年中,许多统计和机器学习算法已应用于交通分类问题。但是,作者使用不同的方法学数据集来评估其方法,因此结果无法直接比较。 大多数方法使用监督或半监督机器学习算法对流进行分类,甚至确定流的应用协议。大多数方法都针对加密协议,例如安全外壳(SSH),安全套接字层(SSL)/传输层安全性(TLS)和加密的BitTorrent。

\subsection{基于协议的HTTPS流量识别}

\subsection{基于机器学习的HTTPS流量识别}

\subsection{基于深度学习的HTTPS流量识别}


\subsection{细粒度的应用识别技术}
\subsection{传统的统计特征应用识别}

\citet{Shbair2016Improving}实现基于SNI给流量添加标签。这篇文章主要的思路是:利用域名数据集( 两个数据集:\href{https://scans.io/series/443-https-tls-alexa_top1mil}{数据集1},\href{https://cs.uwaterloo.ca/ t55wang/wf.html}{数据集2}),使用爬虫工具进行访问,获取到在建立握手的Client hello过程中返回的SNI( 其实是一个扩展extension,名称为server name),为了识别SNI是否是真是的又结合了DNS解析SNI的结果和真实的ip进行比较。
\begin{figure}[!htbp]
	\centering
	\includegraphics[width=0.80\textwidth]{SNI_vertification}
	\bicaption{SNI结合DNS验证}{}
	\label{fig:SNI_vertification}
\end{figure}
\subsection{存在的问题}




\section{论文主要研究内容}

本文主要针对各种应用HTTPS流量进行分析,HTTPS应用流量分分析具有非常重要的意义:

我们的目标之对加密的网络流量进行精细化分类:
\begin{figure}[!htbp]
	\centering
	\includegraphics[width=0.80\textwidth]{detail_classification}
	\bicaption{网络流量精细化分类}{}
	\label{fig:gt_structure}
\end{figure}

\section{论文组织架构}
本文的组织结构如下:

第一章:绪论。在本章中,首先介绍了移动应用数量快速增长的趋势。随后介绍了针对针对移动应用流量分类问题的研究现状。最后概述本文主要的研究内容和本文的组织结构。

第二章:相关研究概述。

第三章:数据集和数据预处理

第四章:基于HTTPS流量的HTTPS流量应用多视图特征识别方法的研究。

第五章:基于增量学习的HTTPS新应用识别。

第六章:总结和展望。本章总结了本文的主要成果,并对本文所提出的应用及其进行了扩展。
