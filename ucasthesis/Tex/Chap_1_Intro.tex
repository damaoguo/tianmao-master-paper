\chapter{绪论}\label{chap:introduction}
本章首先介绍了移动互联网的发展趋势以及HTTPS协议在移动应用中的大规模部署现状,进而引出开展基于HTTPS流量的移动应用识别技术具有重要研究意义。然后调研了当前关于移动应用识别技术的国内外研究进展情况,分析了现有成果的技术原理及其局限性。最后给出了本文的研究目标及研究内容,介绍了本文的组织架构。


\section{研究背景及意义}
移动互联网的迅速发展导致移动应用种类和数量急剧增加,不同的移动应用在人们的日常学习、工作和生活中扮演这着不可或缺的角色,覆盖了咨询、交通、支付、新闻、教育的方方面面,丰富的功能极大改善了人们的生活、提高了生活质量。据中国互联网络信息中心2019年发布的《第43次中国互联网络发展状况统计报告》\citep{cnnic2019}显示,截至2018年12月,中国网民规模已达8.29亿,其中手机网民占比为98.6\%,中国已全面进入移动互联网时代。如图\ref{fig:app-2018}所示,当前仅在中国移动应用市场的移动应用数量就达到449万,并且其在架数量还在持续增长。移动应用的大量使用使得移动互联网流量发生了爆炸性的增加,如图\ref{fig:mobile-traffic},2018年移动互联网接入流量消费达711.1亿,较2017年增长了189.1\%。

\begin{figure}[!htbp]
    \centering
    \begin{subfigure}[b]{0.45\textwidth}
      \includegraphics[width=\textwidth]{APPS-2018.png}
      \caption{}
      \label{fig:app-2018}
    \end{subfigure}%
    ~
    \begin{subfigure}[b]{0.45\textwidth}
      \includegraphics[width=\textwidth]{MobileTraffic.png}
      \caption{}
      \label{fig:mobile-traffic}
    \end{subfigure}
    \bicaption{移动互联网资源应用情况。(a)移动应用在架数量;(b)移动互联网接入流量。}{Mobile Internet Resources.(a)Number of APPs, (b)Mobile Internet Traffic.}
    \label{fig:mobile-network}
\end{figure}



移动互联网流量爆炸增长的同时,为了保证通信的安全性,HTTPS(Hyper Text Transfer Protocol Secure)协议在移动应用中的使用也越来越广泛。根据NSS实验室\footnote{https://www.nsslabs.com/}的报告,截至2019年8月,全网超过58\%的Web应用流量使用了HTTPS\citep{nss2019}。目前各大互联网公司也正在积极推进再其服务和产品推广使用HTTPS承载数据通信,如谷歌公司正在致力于确保其网站和服务默认提供现代HTTPS,它的目标是在其产品和服务上实现100%的加密\citep{google2018google};苹果公司宣布从2017年1月1日起APP Store中的iOS应用必须使用APP Transport Security(ATS)安全功能,要求之后所有上架的IOS应用与服务器之间的通信必须使用TLS版本为1.2及以上的HTTPS\citep{apple2020}。

由于大量的移动互联网流量使用了HTTPS加密技术,这给细粒度的HTTPS流量的精细化识别带来了巨大挑战,其背后主要的原因是传统的用于处理明文流量的深度包检测技术不能够直接迁移到HTTPS加密流量识别的任务场景。因此亟需寻求基于HTTPS流量的移动应用识别技术,如下具体从三个方面介绍其具体的研究必要性:

(1)当前移动应用产生的流量主要包含明文的HTTP流量和加密的HTTPS流量两部分。对于明文HTTP流量进行移动应用识别,主要可利用已知的协议规范信息作为识别特征(如User Agent字段所携带的应用标识信息、Host字段所携带的域名信息等),进而部署基于正则匹配算法完成识别任务。对于HTTPS加密流量的应用识别,仅有少量研究工作关注此部分流量并且其识别效果鲁棒性较差。因此,实现针对HTTPS流量的移动应用识别是移动互联网流量监控、网络服务质量保证、网络资源规划等任务必须解决的核心技术需求。

(2)除了正常的移动应用之外,大量的恶意移动应用也采用了HTTPS加密技术,这些加密流量中是否携带着未知威胁或者存在用户隐私泄露的情况,我们不得而知。因此解决HTTPS流量的应用识别问题对于开展网络空间安全任务是必不可少的基础技术手段。

(3)移动应用识别属于流量分类领域范畴,相比于应用层协议识别和应用类型分类等粗粒度分类任务,是一种更为细粒度和更具挑战的分类任务。图\ref{fig:detail-identification}给出了移动应用识别在整个流量分类研究框架中的位置。

综上所述,本文紧紧围绕基于HTTPS流量的移动应用识别这一技术难题展开深入研究,在充分分析HTTPS加密流量特点的基础上,通过借鉴深度学习、增量学习等技术,实现对HTTPS加密流量与产生其具体应用的关联映射,探索其在解决移动应用识别任务中的可行性和有效技术手段。
\begin{figure}[!htbp]
	\centering
	\includegraphics[width=0.80\textwidth]{Detail-identification.pdf}
	\bicaption{网络流量识别分类研究框架图}{Refined classification of network traffic}
	\label{fig:detail-identification}
\end{figure}

\section{国内外相关研究现状}
当前HTTPS流量识别研究已经取得了一些成果。本章将当前的研究工作归纳为三类:基于规则匹配的方法、基于机器学习的识别方法及基于深度学习的识别方法。

\subsection{基于规则匹配的识别方法}
匹配规则是指应用各种条件来确定新记录或者经过编辑的记录上的字段与现有记录上的相同字段有多高的匹配程度,标准匹配规则是人工预定义条件。在基于规则匹配的识别方法中首先需要对离线的应用流量进行分析,为每一个应用建立匹配规则库。然后当应用新的流量到来时和建立的匹配规则库基于正则表达式、KMP等方法和算法进行匹配,从而完成识别。在使用HTTPS承载网络通信的过程中,SSL/TLS会携带上层应用的标记,比如SNI、Common Name等字段信息,所以HTTPS协议通信过程中依然会泄露一些信息可以被用于完成流量识别任务。因此可以通过这些信息和单个应用特定的、预定义的流量识别条件进行匹配,完成识别任务。

\subsubsection{基于SNI的识别方法}
SNI是TLS ClientHello消息中一个扩展域,表现形式为一个明文字符串,是一个可以进行DNS解析的域名,起到了将TLS请求正确导向服务端的作用。基于SNI的HTTPS流量应用识别技术是一种广泛使用的实用技术,该技术使用SSL/TLS握手的字段---服务器名称指示(Sever Name Identification)来识别HTTPS连接中访问的网站。 基于SNI的监视已集成到许多防火墙解决方案(Sphire wall\footnote{http://www.sphirewall.net},Untangle NG\footnote{https://www.untangle.com},IPFire3\footnote{http://www.ipfire.org}等)中。表~\ref{tab:sni-sample}展示了百度地图和百度贴吧两个应用的HTTPS流中解析出的部分SNI字段。可见,百度贴吧和百度地图两款应用都包含了SNI字段:\emph{gss0.bdstatic.com}和\emph{sofire.baidu.com}。因此使用SNI在进行HTTPS流量精细化分类的任务时候存在混淆,同一个应用厂商的应用程序存在使用相同的SNI的情况。
\begin{table}[!htbp]
    \bicaption{SNI样例。}{SNI examples.}
    \label{tab:sni-sample}
    \centering
    \footnotesize% fontsize
    \setlength{\tabcolsep}{4pt}% column separation
    \renewcommand{\arraystretch}{1.2}%row space 
    \begin{tabular}{cc}
        \hline
        应用名称 & Sever Name Identification(SNI)\\
        \hline
        \multirow{5}{*}{百度地图} & hmma.baidu.com\\
        & gss0.bdstatic.com \\
        & sofire.baidu.com \\
        & api.tuisong.baidu.com\\
        & loc.map.baidu.com\\
        \hline
        \multirow{4}{*}{百度贴吧} & gss0.bdstatic.com\\
        & sofire.baidu.com \\
        & browserkernel.baidu.com \\
        & client.map.baidu.com \\
        \hline
    \end{tabular}
\end{table}

\citet{Shbair2016Improving}提出基于SNI给流量添加标签的方法。这项研究工作的主要思路是:使用爬虫工具进行访问域名数据集\footnote{https://scans.io/series/443-https-tls-alexa\_top1mil}$^{,}$\footnote{https://cs.uwaterloo.ca/t55wang/wf.html},获取到HTTPS通信建立握手的Client hello过程中返回的SNI,为了识别SNI是否是真实的,又结合了DNS解析SNI的结果和真实的IP进行比较,通过IP请求域名服务器,得到该IP所对应的域名,将得到的域名和SNI进行比较,判断该SNI是否正确。这项研究工作还统计了514个HTTPS网站进行了SNI扩展域的支持度,结果表明由230个网站并非按照RFC标准支持的SNI扩展域。基于SNI的技术不能够处理SNI被移除的流,也不能够处理SNI被造假的情况。

\subsubsection{基于数字证书的识别方法}
基于数字证书的识别方法是指根据证书的编码方式对证书进行解析,提取证书的使用者信息,通过规则匹配的方式初步识别判断应用类型的方法。所述的证书是指:在SSL/TLS协议中,X.509证书用于客户端和服务端进行对方身份的验证,其主要包含了域名的公钥,证书权威机构对证书本身的数字签名信息,证书的有效期,以及证书的发行者和拥有着的一些基本信息。在证书中,最为重要的一个属性为证书域名,该属性位于证书的Common Name(CN)域下。和SNI扩展域相同,证书可以用于对HTTPS应用识别任务。

如图\ref{fig:common-name-example}是淘宝应用的一个Common Name的样例,解析出的Common Nmae为\textit{*.alicdn.com},可以将该应用识别为阿里系的应用。但是在实际网路中,存在多个站点使用同一个数字证书的情况,同时很多网站的证书已经过期\citep{holz2011ssl},甚至部分证书并没有经过证书颁发机构(Certificate Authority,CA)签名认证,因此基于Common Name的HTTPS应用识别存在着天然的缺陷。
\begin{figure}[!htbp]
	\centering
	\includegraphics[width=0.70\textwidth]{Wireshark-common-name.png}
	\bicaption{Common Name样例}{Common Name example}
	\label{fig:common-name-example}
\end{figure}


\subsection{基于机器学习的识别方法}
本节所述的机器学习方法是指经典的机器学习方法,如:朴素贝叶斯、支持向量机、随机森林、隐马尔科夫链等。在过去的几十年里,大量研究人员已经探索了机器学习在网络流量识别领域的可行性,并且取得了受业界广泛认可的研究成果。根据所采用的识别特征可以分为基于协议状态转移特征的识别方法和基于流统计特征的识别方法两大类。

\subsubsection{基于协议状态转移特征的识别方法}
对于网络协议而言,协议在不同的阶段有不同的动作,表现为协议交互过程中的不同“状态”,而网络协议的这种有先后顺序的“状态”就是网络流时序特征的准确反映。如在SSL/TLS协议中,消息类型字段随着网络交互的过程而发生转移,反映出一种时序特征,该字段的具体取值和含义见表\ref{tab:message-type}。

\begin{table}[!htbp]
    \bicaption{TLS消息类型。}{TLS Heartbeat Message Type.}
    \label{tab:message-type}
    \centering
    \footnotesize% fontsize
    \setlength{\tabcolsep}{4pt}% column separation
    \renewcommand{\arraystretch}{1.2}%row space 
    \begin{tabular}{rccc}
    \hline
    \textbf{取值} & \textbf{描述} & \textbf{DTLS-OJ} & \textbf{参考}\\
    \hline
    0 & Reserved & & [RFC6520]\\
    1 &	heartbeat\_request & Y & [RFC6520]\\
    2 &	heartbeat\_response & Y & [RFC6520]\\
    3-254 &	Unassigned & & \\		
    255 & Reserved & & [RFC6520]\\
    \hline
    \end{tabular}
\end{table}


\citet{korczynski2014markov}采用SS/TLS协议交互过程种的消息类型序列建立一阶马尔可夫链模型,用于识别加密应用。但是由于该模型只考虑非常有限的离散转移状态且不同的应用之间转移状态重叠性高,该方法在某些情况下会失效。

\citet{shen2017classification}利用TLS嵌入的消息类型作为状态,并在数据包级别上建立了二阶隐马尔可夫模型,扩展了应用程序的指纹。


\subsubsection{基于流统计特征的识别方法}
当前基于统计特征的流量识别方法的非常丰富,针对HTTP流量,\citet{Moore2005Internet}在2005年提出提出了248中特征并使用朴素贝叶斯算法进行网络流量分类。这248中特征具有普适性,能够迁移到处理各种协议的分析工作。这些特征主要包括:数据包个数,数据包头部大小,TCP分片大小,数据包到达的时间等时间和大小上的特征,包括中位数、众数、均值、方差、偏差、极差等统计量。这项研究工作是后续基于统计特征流量识别工作的基础,为流量特征工程提供了重要参考。

当前,HTTPS流量识别的现有技术通也常采用基于统计功能的机器学习方法\citep{Muehlstein2016Analyzing, Kohout2017Network}。\citet{Wright2006On}提出了一种基于大小和时间的特征提取方法,使用隐马尔科夫模型完成识别任务,该研究工作的识别粒度为协议层分类(SSH,HTTPS以及非加密流量)。实验结果表明不同的网络应用的数据包的长度差异很大。 \citet{Fahad2013Toward}提出了一种选择流量特征的方法,该方法通过分析协议交互过程总结出网络流量中各种应用协议最佳的特征属性。 \citet{Muehlstein2016Analyzing}对加密网络流量的统计分析,识别用户的操作系统,浏览器和应用程序。这些基于统计特征的方法依赖于复杂的特征构造和提取。

机器学习给流量识别领域大量新的活力,但是依然存在一定的缺陷,主要体现在:特征设计和选择的复杂,当网络环境变化时通用性较差,在本文的第四章节将提取流统计特征并使用机器学习进行实验和本文中的方法进行对比。

\subsection{基于深度学习的识别方法}
深度学习通过自动提取选择特征在一定程度上消除了对专家知识的依赖,可以处理当新类别不断涌现并且旧类别的模式不断演变的情况,与传统的机器学习方法相比,它具有更高的学习能力,因此可以学习高度复杂的模式。当前深度学习图像、自然语言处理、语音处理等多个领域取得丰硕的成果,越来越多的研究人员开始使用深度学习方法处理HTTPS流量识别相关的问题。

\citet{Wang2017End}提出了一种端到端流量分类方法,该方法提取忽略任何其他信息的流或会话的前784个字节,然后使用一维卷积进行处理提取抽象特征。该模型在一个12个应用的加密数据集上进行了评估,并显示出比使用时间序列和统计功能的C4.5方法有显着提升。这项工作没有考虑到HTTPS协议相关知识,在实现端到端的识别能力的同时牺牲了一定的识别能力,本文将这项工作作为Baseline处理本文的流量数据集,在第四章和本文提出的基于多视图特征的流量识别方法进行对比。

\citet{lotfollahi2017deep}提出Deep packet方法处理流量识别问题,该方法采用堆叠自动编码器(SAE)和卷积神经网络(CNN)两种网络结构对网络流量进行分类。该方法既可以处理将网络流量分为主要类别(例如FTP和P2P)的流量表征,也可以处理需要识别最终用户应用程序(例如BitTorrent和Skype)的应用识别任务。但是针对应用数量持续涌现的情况,并没有给出很好的解决方法。


\citet{lopez2017network}提出一种混合卷积神经网络和循环神经网络的流量识别方法,该采用CNN模型处理流的前6至30个数据包的内容,然后输入给RNN或LSTM模型。实验结果表明这种混合模型的方法要优于单纯使用一种深度学习模型的方法。本文中提出的方法也正是混合了卷积神经网络和循环神经网络来提取抽象特征,完成识别任务。

\citet{wang2020real}首先开发了一款Android应用程序工具,称作为NetLog,用于捕获来自智能手机的所有流量,并按应用程序名称准确标记流量。然后使用该工具构建了一个由142个应用组成的大型私有移动流量数据集,包括相当多的未加密和加密流量,用于分类评估。接着将深度学习技术应用于APP-ID任务,并在数据集中评估三个不同的分类器:堆叠式降噪自动编码器(SDAE),卷积神经网络和长短期记忆网络。最后对其提出的模型的可解释性进行了分析。


\section{研究目标及内容}
移动互联网的快速发展使得移动互联网流量爆炸增长,针对移动互联网流量的精细化分类是网络安全空间和网络管理的核心技术。处于通信安全和隐私保护的目的,越来越多的移动应用采用HTTPS协议作为数据承载,这给移动应用流量识别带来了巨大的挑战。当前HTTPS应用识别工作存在数据标记粗糙,特征建模难,针对新增应用存在灾难遗忘问题。本文旨在研究精细化的HTTPS网络应用特征提取与识别方法,为HTTPS流量精细化分类提供有效的解决方法。本文所提出的方案主要包括:自动化移动应用HTTPS流量标记方法、基于多视图序列特征的应用识别方法、基于增量学习的新增应用识别方法。

\subsection{自动化移动应用HTTPS流量标记方法}
当前HTTPS流量数据分类相关的研究工作主要依赖于私有的数据集,公开的、标记精细的、有效的HTTPS流量数据有限,这是当前研究HTTPS流量精细化分类任务的主要障碍。在本文中,首先提出一种自动化的流量数据标记方法,主要解决由于标记粗糙导致的识别粒度粗的问题。

\subsection{基于多视图序列特征的移动应用识别方法}
针对传统的特征工程结合机器学习的识别方法存在的特征定义选择困难,依赖特征工程高,特征建模能力弱的问题,以及当前很多使用深度学习的方法解决HTTPS流量精细化分类任务的工作作一味追求深度学习的端到端方案,忽视HTTPS协议本身定义(即完全不考虑容易获取的专家知识的作用)的现状,本文从HTTPS协议本身的定义和通信过程出发,提出了一种基于多视图特征的HTTPS流量精细化分类方法,多视图特征由负载、数据包大小和HTTPS通信过程中每个数据包的都会携带的标记---内容类型三个部分构成。针对三个视图特征采用深度学习的方法提取抽象特征,由三个视图提取的抽象特征共同完成识别任务。

\subsection{基于增量学习的新增移动应用识别方法}
随着移动互联网的使用越来越广泛,移动应用的数量持续增加,这意味着识别模型需要不断地进行更新以完成针对新应用的识别,深度学习在面对应用增加的时候不可避免地需要面对灾难遗忘问题。为了有效处理新应用地识别问题,本文提出基于增量学习地新应用识别方法,当新的应用流量数据到达的时候,该方法综合使用旧模型,旧数据和新增流量数据共同训练识别模型,并且有效控制模型的空间和时间复杂度。

\section{论文组织架构}
围绕上述的研究目和研究内容,本论文由六章构成,论文的组织架构如图\ref{fig:paper-architecture}所示,具体如下:

\begin{figure}[!htbp]
	\centering
	\includegraphics[width=0.80\textwidth]{Paper-architecture.pdf}
	\bicaption{论文组织结构图}{Paper architecture}
	\label{fig:paper-architecture}
\end{figure}

第一章:绪论。在本章中,首先介绍了移动应用数量快速增长的趋势。随后从基于协议字段规则匹配的方法、基于机器学习的识别方法和基于深度学习的识别方法三个方面介绍了针对移动应用流量分类问题的研究现状,最后概述本文主要的研究内容和本文的组织结构。

第二章:相关概念及技术。本章首先对HTTPS协议进行了研究,重点关注了HTTPS握手过程过程以及协议本身定义的属性信息。接着本章介绍在本研究工作中使用到的机器学习、深度学习、增量学习方法,对不同的模型进行了阐述并调研了相关的研究工作。

第三章:自动化移动应用HTTPS流量标记方法。首先调研了现在存在的公开数据集,对不同的数据集进行了分析,并分析在本课题中公开的数据集不能够使用的原因。接着详细介绍了精细化的流量数据采集标定方法,并构建了精细化标记的HTTPS流量数据集,为后续的工作提供数据支撑。

第四章:基于多视图特征的移动应用识别方法。本章首先负载、内容类型以及数据包大小进行分析,确定用于流量识别的视图。然后使用深度学习的方法从三个视图中提取抽象的特征,训练识别模型。最后针对20种应用进行实验,验证本章提出的AIBMF模型的有效性。

第五章:基于增量学习的新增移动应用识别方法。本章首先分析了因为新增应用带来模型对旧应用识别能力的丧失的问题。在真实的场景中,移动应用在不断出现和更新,然而深度学习的方法在处理类别增加的问题的时候,会面临“灾难遗忘”的问题,导致模型的识别能力快速下降。针对灾难遗忘问题本章接着提出了InceAIBMF模型,结合新增应用流量、旧应用流量和旧模型处理真实场景中新增应用的问题。最后对IncreAIBMF模型的效果进行了评估,并重点关注该模型在训练时的空间和时间消耗。

第六章:总结和展望。本章总结了本文的主要成果,并对可以继续开展的工作进行展望。
