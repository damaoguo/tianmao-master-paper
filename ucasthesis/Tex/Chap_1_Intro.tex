\chapter{绪论}\label{chap:introduction}
本章从移动互联网的迅速发展和HTTPS协议的广泛使用的现状引出HTTPS流量的应用识别的重要意义,然后介绍了HTTPS流量识别技术发展的历史和现状,分析了当前主要的技术手段和方法的效果以及存在的不足,最后概括了本文的组织架构和相关工作。


\section{研究背景及意义}
移动互联网的迅速发展导致移动应用数量(种类)急剧增加,不同的移动应用在人们的日常学习、工作和生活中扮演这着不可或缺的角色,覆盖了咨询、交通、支付、新闻、教育的方方面面,丰富的功能极大改善了人们的生活、提高了生活质量。据中国互联网络信息中心2019年发布的《第43次中国互联网络发展状况统计报告》显示,
截至2018年12月,中国网民规模已达8.29亿,其中手机网民占比为98.6\%,中国已全面进入移动互联网时代。当前仅在中国的应用市场,移动应用程序(Apps)数量就有449万\citep{cnnic2019}。

移动应用的大量使用使得移动应用流量也发生了爆炸性的增加,为了保证通信的安全性,HTTPS(Hyper Text Transfer Protocol Secure)协议在移动应用中的使用也越来越广泛。NSS实验室\footnote{https://www.nsslabs.com/}检测到:截至2019年8月,全网超过58\%的web流量使用HTTPS传输\citep{nss2019}。Barac检测到当前超过90\%的流量数据已经被加密\citep{barac2019}。谷歌公司正在致力于确保其网站和服务默认提供现代HTTPS。它的目标是在其产品和服务上实现100%的加密\citep{google2018google}。2016年6月,苹果公司宣布从2017年1月1日起APP Store中的IOS应用必须使用APP Transport Security(ATS)安全功能,要求之后所有上架的IOS应用与服务器之间的通信必须使用TLS版本为1.2及以上的HTTPS\citep{apple2020}。

针对HTTPS流的精细化识别(\emph{将加密的HTTPS流与特定的移动应用程序关联})是流量监控,QoS保证,网络异常检测等任务场景中的基本技术,通过对移动流量的识别,有效分配网络资源,监控移动应用流量的使用分布情况,网络管理员、运营商以及相关的企业可以制定不同的策略,对网络有效管理。通过HTTPS流的移动应用进行识别,可以发现异常的应用程序,有效识别使用HTTPS流量的恶意软件(如病毒、木马等),对恶意应用产生的流量进行隔绝,保障用户的隐私和财产安全。针对移动互联网的流量的分析工作越来越重要。

本文的目标是对加密的网络流量进行精细化分类,精细化识别结构如图\ref{fig:detail-identification},精细化识别的识别粒度更加精细,是将单条HTTPS流和产生它的应用进行关联。
\begin{figure}[!htbp]
	\centering
	\includegraphics[width=0.80\textwidth]{Detail-identification.pdf}
	\bicaption{网络流量精细化分类}{Refined classification of network traffic}
	\label{fig:detail-identification}
\end{figure}

综上所述,基于HTTPS流量的移动应用识别技术在网络安全中有着重要的作用。随着移动应用的不断涌现以及HTTPS协议在移动应用上的不断推广,对于流量进行精细化识别的技术方法面临着更大的挑战。本文研究使用深度学习、增量学习的方法依靠多视图特征更加高效、准确地将HTTPS流和应用进行关联,在本文中所提到的对流量进行识别即为对HTTPS流量进行精细化分类,即识别和分类为同一个概念。


\section{国内外相关技术及其研究现状}
当前HTTPS流量识别研究已经取得了一些成果,针对HTTPS流量识别的任务首要是根据应用需求确定识别对象和识别粒度,根据识别对象和识别粒度选择合适的方法。本章根据流量识别方法将当前的研究工作归纳为三类:基于规则匹配的方法、基于机器学习的识别方法及基于深度学习的识别方法。

\subsection{基于规则匹配的识别方法}
匹配规则是指应用各种条件来确定新记录或者经过编辑的记录上的字段与现有记录上的相同字段有多高的匹配程度,标准匹配规则是人工预定义条件。加密协议会对上层应用进行标记,比如SNI、Common Name等字段信息,所以HTTPS协议通信过程中依然会泄露一些信息可以被用于完成流量识别任务。通过这些信息和单个应用特定的、预定义的流量识别条件进行匹配,完成识别任务。

\subsubsection{基于SNI的HTTPS流量应用识别}
SNI是TLS ClientHello消息中的明文字符串值,它提供了一种方便的方法来了解新HTTPS连接所访问的服务。基于SNI的HTTPS流量应用识别技术是一种广泛使用的实用技术,该技术使用SSL/TLS握手的字段---服务器名称指示(Sever Name Identification)来识别HTTPS连接中访问的网站。 基于SNI的监视已集成到许多防火墙解决方案(Sphire wall\footnote{http://www.sphirewall.net},Untangle NG\footnote{https://www.untangle.com},IPFire3\footnote{http://www.ipfire.org}等)中。表~\ref{tab:sni-sample}展示了百度地图、百度贴吧、淘宝三个应用的HTTPS流中解析出的部分SNI字段。可见,百度贴吧和百度地图两款应用都包含了SNI字段:\emph{gss0.bdstatic.com}和\emph{sofire.baidu.com}。因此使用SNI在进行HTTPS流量精细化分类的任务时候存在混淆,同一个应用厂商的应用程序存在使用相同的SNI的情况。
\begin{table}[!htbp]
    \bicaption{SNI样例。}{SNI examples.}
    \label{tab:sni-sample}
    \centering
    \footnotesize% fontsize
    \setlength{\tabcolsep}{4pt}% column separation
    \renewcommand{\arraystretch}{1.2}%row space 
    \begin{tabular}{cc}
        \hline
        应用名称 & Sever Name Identification(SNI)\\
        \hline
        \multirow{5}{*}{百度地图} & hmma.baidu.com\\
        & gss0.bdstatic.com \\
        & sofire.baidu.com \\
        & api.tuisong.baidu.com\\
        & loc.map.baidu.com\\
        \hline
        \multirow{4}{*}{百度贴吧} & gss0.bdstatic.com\\
        & sofire.baidu.com \\
        & browserkernel.baidu.com \\
        & client.map.baidu.com \\
        \hline
        \multirow{4}{*}{淘宝} & nbsdk-baichuan.alicdn.com \\
        & dtmall-tel.alicdn.com\\
        & dorangesource.alicdn.com\\
        & umdc.aliapp.org\\
        \hline
    \end{tabular}
\end{table}


% ......提出了.......该方法(思路,使用了..........步骤是.......),最后实验评估.........

% 优点......
% 缺点......



\citet{Shbair2016Improving}提出基于SNI给流量添加标签的方法。这项研究工作的主要思路是:使用爬虫工具进行访问域名数据集(\footnote{https://scans.io/series/443-https-tls-alexa\_top1mil},\footnote{https://cs.uwaterloo.ca/t55wang/wf.html},获取到HTTPS通信建立握手的Client hello过程中返回的SNI,为了识别SNI是否是真实的,又结合了DNS解析SNI的结果和真实的IP进行比较,通过IP请求域名服务器,得到该IP所对应的域名,将得到的域名和SNI进行比较,判断该SNI是否正确。基于SNI的技术不能够处理SNI被移除的流,也不能够处理SNI被造假的情况。

\subsubsection{基于数字证书的识别方法}
基于数字证书的识别方法是指根据证书的编码方式对证书进行解析,提取证书的使用者信息,通过规则匹配的方式初步识别判断应用类型的方法。所述的证书是指:在SSL/TLS协议中,X.509证书用于客户端和服务端进行对方身份的验证,其主要包含了域名的公钥,证书权威机构对证书本身的数字签名信息,证书的有效期,以及证书的发行者和拥有着的一些基本信息。在证书中,最为重要的一个属性为证书域名,该属性位于证书的Common Name(CN)域下。和SNI扩展域相同,证书可以用于对HTTPS应用识别任务。如图\ref{fig:common-name-example}是淘宝应用的一个Common Name的样例,解析出的Common Nmae为\textit{*.alicdn.com},可以将该应用识别为阿里系的应用。但是在实际网路中,存在多个站点使用同一个数字证书的情况,同时很多网站的证书已经过期\citep{holz2011ssl},甚至部分证书并没有经过证书颁发机构(Certificate Authority,CA)签名认证,因此基于CN的HTTPS应用识别存在着天然的缺陷。
\begin{figure}[!htbp]
	\centering
	\includegraphics[width=0.80\textwidth]{Wireshark-common-name.png}
	\bicaption{Common Name样例}{Common Name example}
	\label{fig:common-name-example}
\end{figure}


\subsection{基于机器学习的识别方法}
本节所述的机器学习方法是指经典的机器学习方法,如:朴素贝叶斯、支持向量机、随机森林、隐马尔科夫链等。在过去的几十年里,大量研究人员已经探索了机器学习在网络流量识别领域的可行性,并且取得了受业界广泛认可的研究成果。基于机器学习的识别方法可以分为基于协议状态转移特征的识别方法和基于流统计特征的识别方法两大类。

\subsubsection{基于协议属性信息的识别方法}
\citet{shen2017classification}利用TLS嵌入的消息类型作为状态,并在数据包级别上建立了隐马尔可夫模型,扩展了应用程序的指纹,验证数据包的消息类型功能可用于流量识别。消息类型字段的具体取值和含义见表\ref{tab:message-type}。

\begin{table}[!htbp]
    \bicaption{TLS消息类型。}{TLS Heartbeat Message Type.}
    \label{tab:message-type}
    \centering
    \footnotesize% fontsize
    \setlength{\tabcolsep}{4pt}% column separation
    \renewcommand{\arraystretch}{1.2}%row space 
    \begin{tabular}{rccc}
    \hline
    \textbf{取值} & \textbf{描述} & \textbf{DTLS-OJ} & \textbf{参考}\\
    \hline
    0 & Reserved & & [RFC6520]\\
    1 &	heartbeat\_request & Y & [RFC6520]\\
    2 &	heartbeat\_response & Y & [RFC6520]\\
    3-254 &	Unassigned & & \\		
    255 & Reserved & & [RFC6520]\\
    \hline
    \end{tabular}
\end{table}
分析可得,消息类型主要用于心跳响应中,在HTTPS流中,并非每一条流或者流中每一个包都会被消息类型的值标记,因此对于不被消息类型标记的流,这种方法将会失效。


\subsubsection{基于流统计特征的识别方法}
当前基于统计特征的流量识别方法的非常丰富,针对HTTP流量,\citet{Moore2005Internet}在2005年提出使用机器学习的方法进行网络流量分类,并且提出了248中特征。这248中特征具有普适性,能够迁移到处理各种协议的分析工作。这些特征主要包括:数据包个数,数据包头部大小,TCP分片大小,数据包到达的时间等时间和大小上的特征,并计算中位数、众数、均值、方差、偏差、极差等统计量,计算熵以及傅里叶变换等。同时,这项工作还对特种工程得到的特征的重要性进行了评估。HTTPS流量识别的现有技术通常采用基于统计功能的机器学习方法\citep{Muehlstein2016Analyzing, Kohout2017Network}。\citet{Wright2006On}提出了一种基于大小和时间的特征提取方法。 Fahad等人\citep{Fahad2013Toward}提出了一种选择流量特征的方法,其主要目的是获得最重要的流量特征。 Jonathan等人\citep{Muehlstein2016Analyzing}展示的方法专注于加密网络流量的统计分析,以识别用户的操作系统,浏览器和应用程序。它们提供了利用浏览器的突发行为和SSL(安全套接字层)行为的新功能。这些基于统计特征的方法依赖于复杂的特征构造和提取。

基于统计特征的HTTPS流量精细化识别的主要缺点是特征设计和选择的复杂,当网络环境变化时通用性较差,在本文的第四章节将对比使用机器学习进行实验。

\subsection{基于深度学习的识别方法}
当前深度学习图像、自然语言处理、语音处理等多个领域取得丰硕的成果,越来越多的研究人员开始使用深度学习方法处理HTTPS流量识别相关的问题。

\citet{Wang2017End}提出了一种端到端流量分类方法,该方法提取忽略任何其他信息的流或会话的前784个字节,然后使用一维卷积进行处理,这项工作没有考虑到HTTPS协议相关知识,在实现端到端的识别能力的同时牺牲了一定的识别能力,本文在第四章节将会对其进行比较。

\citet{lotfollahi2017deep}提出了一种基于深度学习的加密流量分类的方法---Deep packet,该方法采用两种深度神经网络结构,即堆叠自动编码器(SAE)和卷积神经网络(CNN),以便对网络流量进行分类。该方法既可以处理将网络流量分为主要类别(例如FTP和P2P)的流量表征,也可以处理需要识别最终用户应用程序(例如BitTorrent和Skype)的应用程序识别。但是针对应用数量持续涌现的情况,并没有给出很好的解决方法。


\section{研究目标及内容}
针对当前HTTPS应用识别存在的依赖特征工程高、特征建模能力弱、识别粒度粗、网路流量特征和分布受时间和网络状态影响大,针对新增应用存在灾难遗忘问题,本文旨在研究精细化的HTTPS网络应用特征提取与识别方法,为HTTPS流量精细化分类模型的稳定和高效提供一种有效的解决方法。本文所提出的方案主要包括:自动化流量数据标记方法、基于多视图序列特征的应用识别方法、基于增量学习的新应用识别方法。

\subsection{自动化移动应用HTTPS流量标记方法}
当前HTTPS流量数据分类相关的研究工作主要依赖于私有的数据集,公开的、标记精细的、有效的HTTPS流量数据有限,这是当前研究HTTPS流量精细化分类任务的主要障碍。在本文中,首先提出一种自动化的流量数据标记方法,主要解决由于标记粗糙导致的识别粒度粗的问题。

\subsection{基于多视图序列特征的移动应用识别方法}
针对传统的特征工程结合机器学习的识别方法存在的特征定义选择困难,依赖特征工程高,特征建模能力弱的问题,以及当前很多使用深度学习的方法解决HTTPS流量精细化分类任务的工作作一味追求深度学习的end-to-end方案,忽视HTTPS协议本身定义(即完全不考虑容易获取的专家知识的作用)的现状,本文从HTTPS协议本身的定义和通信过程出发,提出了一种基于多视图特征的HTTPS流量精细化分类方法,多视图特征由负载、数据包大小和HTTPS通信过程中每个packet的都会携带的标记---内容类型三个部分构成。针对三个视图特征采用深度学习的方法提取抽象特征,由三个视图提取的抽象特征共同完成识别任务。

\subsection{基于增量学习的新增移动应用识别方法}
随着移动互联网的使用越来越广泛,移动应用的数量持续增加,这意味着识别模型需要不断地进行更新以完成针对新应用的识别,深度学习在面对应用增加的时候不可避免地需要面对灾难遗忘问题。为了有效处理新应用地识别问题,本文提出基于增量学习地新应用识别方法,当新的应用流量数据到达的时候,该方法综合使用旧模型,旧数据和新增流量数据共同训练识别模型,并且有效控制模型的空间和时间复杂度。

\section{论文组织架构}
围绕上述的研究目和研究内容,本论文由六章构成,论文的组织架构如图:\ref{fig:paper-architecture}所示,具体如下:

\begin{figure}[!htbp]
	\centering
	\includegraphics[width=0.80\textwidth]{Paper-architecture.pdf}
	\bicaption{论文组织结构图}{Paper architecture}
	\label{fig:paper-architecture}
\end{figure}

第一章:绪论。在本章中,首先介绍了移动应用数量快速增长的趋势。随后从基于协议字段规则匹配的方法、基于机器学习的识别方法和基于深度学习的识别方法三个方面介绍了针对移动应用流量分类问题的研究现状,最后概述本文主要的研究内容和本文的组织结构。

第二章:相关概念及技术。本章首先对HTTPS协议进行了研究,重点关注了HTTPS握手过程过程以及协议本身定义的属性信息。接着本章介绍在本研究工作中使用到的机器学习、深度学习、增量学习方法,对不同的模型进行了阐述并调研了相关的研究工作。

第三章:自动化移动应用HTTPS流量标记方法。首先调研了现在存在的公开数据集,对不同的数据集进行了分析,并分析在本课题中公开的数据集不能够使用的原因。本章接着详细介绍了精细化的流量数据采集标定方法,并构建了针对Android应用的流量数据。

第四章:基于多视图特征的移动应用识别方法。从三个不同的视图方向进行研究,使用深度学习的方法提取抽象的特征,训练识别模型,针对20种应用进行实验,验证本章提出的AIBMF模型的有效性。

第五章:基于增量学习的新增移动应用识别方法。在真实的场景中,移动应用的数目在不断出现和更新,然而深度学习的方法在处理类别增加的问题的时候,会面临“灾难遗忘”的问题,导致模型的识别能力快速下降。本章提出了InceAIBMF模型用于针对新应用持续增加的场景。并重点关注该模型在训练时候的性能消耗,以及时间消耗。

第六章:总结和展望。本章总结了本文的主要成果,并对本文所提出的应用及其进行了扩展。
