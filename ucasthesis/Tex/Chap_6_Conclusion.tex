\chapter{总结与展望}\label{chap:conclusion}
本章对论文工作和取得的成果进行总结,并展望后续在移动应用HTTPS流量精细化识别任务中需要继续研究的新的研究课题。

\section{总结}
本文的研究目的是研究一种能够快速、准确对移动应用HTTPS流量进行识别的方法,不进行复杂的特征工程,不借助专业的分析工具,不需要流量分析的经验知识。通过总结传统的流量识别方法,本文提出了基于多视图特征的移动应用HTTPS流量识别方法(AIBMF),在这个方法的基础之上,考虑到实际的应用场景,即移动应用更新层出不穷,提出了基于增量学习的移动应用HTTPS流量识别方法(IncreAIBMF)。
本文的主要工作在于:

1. 提出了自动化移动应用HTTPS流量标记方法,并抓取50个移动应用,20多万条流的HTTPS流量数据,是本文研究工作的基础。

2. 提出了基于多视图特征的移动应用识别方法---AIBMF模型,本文首次将针对HTTPS的经验知识(协议本身的定义等)和深度学习方法结合处理流量精细化识别任务,针对不同的视图特征采用恰当的深度学习模型,更加有效提抽象特征。

3. 提出了基于增量学习的新移动应用识别方法---IncreAIBMF模型,解决新应用不断涌现带来的因为灾难遗忘而无法有效识别的问题,在真实的应用场景中对处理新应用流量识别问题具有重要的实用价值。

\section{展望}

综上讨论,本文认为今后的研究工作可以从以下几个方面展开。


1. 在本文中,通过一系列的实验证明了所使用的AIBMF和IncreAIBMF方法在移动应用的HTTPS流量识别任务上可以取得优秀的识别效果。在今后的工作中,可以针对流量的采集工作进行进一步的研究。本文在采集数据时是通过遍历移动应用中的事件产生随机事件产生流量进行捕获的,可以研究使用深度学习、强化学习驱动应用的运行,虽然这个研究方向属于自动化测试领域的工作,但是却在流量数据采集中具有重要的意义,可以更加贴近用户的使用习惯产生应用流量。

2. 本文中提出的AIBMF模型中使用到的HTTPS协议定义的参数只包括内容类型(Content Type),可以探索更多的协议定义的参数在流量精细化识别工作中的价值,结合HTTPS协议和深度学习的方法在流量精细化识别任务上具有很广的探究前景。