\chapter{总结与展望}\label{chap:conclusion}
本章对论文工作和取得的成果进行总结,并展望后续基于HTTPS流量的移动应用识别任务中需要继续研究的新的研究课题。

\section{总结}
本文的研究目的是研究一种能够准确对移动应用HTTPS流量进行识别的方法,减少对复杂的特征工程的依赖,提高移动应用识别的性能。为此提出了基于多视图特征的移动应用HTTPS流量识别方法,在这个方法的基础之上,考虑到实际的应用场景,即移动应用新增以及更新迅速,提出了基于增量学习的新增移动应用识别方法。本文的主要工作在于:

1. 提出了自动化移动应用HTTPS流量标记方法,并抓取50个移动应用,20多万条流的HTTPS流量数据,解决当前所公布的开源数据较少、数据量有限、流量数据标记较为粗糙的问题,为本文后续的研究工作提供了可靠的有标记的实验数据集。

2. 提出了基于多视图特征的移动应用识别方法,首次从多个视图角度对HTTPS流量进行特征建模形成多视图特征,并在此基础上使用深度学习模型进行高阶特征学习和融合,最后完成移动应用识别任务。解决了现有的移动应用识别方法特征建模难、特征建模能力弱、识别模型受网络环境影响高的问题。

3. 提出了基于增量学习的新增移动应用识别方法,借助增量学习思想,利用新增流量数据和代表性流量样本共同训练识别模型,解决新应用不断涌现带来的因为灾难遗忘是的模型无法有效识别的问题,在真实的应用场景中对处理新增应用流量识别问题具有重要的实用价值。

\section{展望}
综上讨论,本文认为今后的研究工作可以从以下几个方面展开。

1. 为了采集更高质量的流量数据,可以针对流量的采集工作进行进一步的研究。本文在采集数据时是通过遍历移动应用中的事件产生随机事件产生流量进行捕获的,可以研究使用深度学习、强化学习驱动应用的运行,借助自动化测试领域的相关研究工作,对于流量数据采集具有重要的意义,可以更加贴近用户的使用习惯产生应用流量。

2. 本文中提出的识别模型中使用到的HTTPS协议定义的参数只包括内容类型(Content Type),可以探索更多的协议定义的参数在流量精细化识别工作中的价值,结合HTTPS协议和深度学习的方法在流量精细化识别任务上具有很广的探究前景。

