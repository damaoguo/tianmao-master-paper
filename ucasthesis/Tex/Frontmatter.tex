%---------------------------------------------------------------------------%
%->> Frontmatter
%---------------------------------------------------------------------------%
%-
%-> 生成封面
%-
\maketitle% 生成中文封面
\MAKETITLE% 生成英文封面
%-
%-> 作者声明
%-
\makedeclaration% 生成声明页
%-
%-> 中文摘要
%-
\intobmk\chapter*{摘\quad 要}% 显示在书签但不显示在目录
\setcounter{page}{1}% 开始页码
\pagenumbering{Roman}% 页码符号
随着移动互联网的快速发展,人们生活和工作已经离不开移动智能设备,相应的移动应用程序爆炸性增长。为了实现智能化运营管理、保证通信质量、净化网络环境和维护网络安全,针对移动应用网络流量的精细化分类已经成为网络空间安全和网络管理的基础核心技术。当前出于通信安全和保护隐私的目的,越来越多的移动应用使用HTTPS协议作为数据承载。HTTPS的广泛使用给传统的流量分析识别技术带来巨大的挑战,特别是基于负载流量和统计特征的识别方法无法有效完成HTTPS加密流量的精细化分类任务。针对当前针对基于HTTPS流量的应用精细化分类存在的特征建模难,准确率低,鲁棒性差的问题,本文的研究工作围绕着精细化的HTTPS流量识别问题进行展开,具体的研究工作包括:

1. 针对当前公共的流量数据集粗糙,流量采集自动化程度低问题,提出了一种自动化移动应用HTTPS流量的标记方法,构建数据集。在使用深度学习进行流量精细化识别任务的时候,需要依赖大量的数据训练识别模型,本文分两个批次构建了一个50个应用共计20多万条流的HTTPS流量数据集,为后续的研究工作奠定基础。

2. 针对当前HTTPS应用特征建模难的问题,提出一种基于多视图特征的移动应用识别方法---AIBMF\footnote{Mobile \underline{A}pplication \underline{I}dentification Over HTTPS Traffic \underline{B}ased on \underline{M}ulti-view \underline{F}eatures},这是一种按应用程序类型对HTTPS流量进行分类的细粒度方法。AIBMF的关键思想是结合三种视图特征---流量负载,数据包大小和内容类型。 融合三种特征,分别针对性使用卷积神经网络和循环神经网络构建了用于流量精细化识别任务的深度学习模型。为了评估AIBMF的有效性,本文在一个真实的数据集(大约100,000个以上的流)上进行了一组全面的实验,结果表明该方法达到了96.06\%的准确度,并且在F1值上优于最新方法3.6\%。

3. 针对移动应用持续爆炸性增长,而旧模型存在\textit{灾难遗忘}的问题,在AIBMF模型的基础上,提出了一种基于增量学习的新增移动应用识别方法,称之为IncreAIBMF\footnote{\underline{Incre}mental Learning for \underline{A}pplication \underline{I}dentification Over HTTPS Traffic \underline{B}ased on \underline{M}ulti-view \underline{F}eatures}。当前针对加密流量识别方法具有一个主要缺点,那就是新应用程序继续遭受\textit{灾难遗忘}的困扰,当逐步增加新的应用程序类别进行培训时,整体性能会急剧下降。这是由于当前模型需要整个数据集(包括来自旧类和新类的所有样本)来更新模型。随着应用程序数量的增长,更新要求变得不可持续,为了解决这个问题,本文提出\emph{IncreAIBMF}框架,使用新的应用程序数据以及仅对应于旧应用程序样本的一小样本集逐步学习深度神经网络。\emph{IncreAIBMF}背后的关键思想是一个增量学习框架,它通过结合交叉精馏损失而拥有新的应用程序识别能力,该学习框架不仅可以学习新的应用程序类别,而且可以保留与旧的应用程序类别相对应的先前知识。实验结果表明,\emph{IncreAIBMF}在包含50个移动应用程序的真实HTTPS流量数据中,分别在宏精度上达到87.3\%,在F1得分上达到87.8\%,在宏调用上达到88.9\%。


\keywords{HTTPS流量,移动应用识别, 多视图特征,深度学习,增量学习}% 中文关键词
%-
%-> 英文摘要
%-
\intobmk\chapter*{Abstract}% 显示在书签但不显示在目录
With the rapid development of the mobile Internet, mobile devices are everywhere in people's daily life and work, and the mobile applications have exploded. In order to intelligently manage network, ensure QOS, purify the network environment, maintain network security, the refined classification of network traffic has become an indispensable part and a basic technology in the field of network security and management. For communication security and privacy protection, more and more mobile applications use the HTTPS protocol as the default communication protocol. The HTTPS protocol uses encryption technology to establish a secure channel between the client and the server to avoid being illegal  Monitor, hijack and tamper during the retransmission of communication data. The wide use of HTTPS brings great challenges to traditional traffic analysis technologies. The method based on load traffic and statistical characteristics can't effectively complete the refined classification task of HTTPS encrypted traffic, and the current mobile application HTTPS traffic refined identification are faced with feature construction The difficulty of modeling, low accuracy, and poor robustness. The research This paper is centered on these issues. The specific research work includes:

1. In-depth research on the HTTPS protocol and explore the view of application identification based on HTTPS traffic. At present, deep learning has achieved excellent results in the fields of images, natural language processing, and speech, which also brings new opportunities for mobile application network traffic recognition technology. This article deeply analyzes the performance of deep learning in different scenarios, and uses classic deep learning models to process HTTPS traffic data.

2. Aiming at the problem of rough public traffic data set, construct a data set. When using deep learning for traffic refinement recognition tasks, a large amount of data is needed to train the recognition model. This research work builds an HTTPS traffic data set of 50 applications totaling more than 200,000 flows in two batches.

3. Aiming at the difficulty of current HTTPS application feature modeling, a multi-view HTTPS mobile application recognition method based on deep learning--AIBMF is proposed. It's a fine-grained method for classifying HTTPS traffic by application type. AIBMF's key idea is to combine three view features-payload convolution (payload), packet size sequence and packet content type sequence. Based on these different view features, a deep learning model for HTTPS traffic recognition tasks is constructed. In order to evaluate the effectiveness of AIBMF, this paper performed a comprehensive set of experiments on a real data set (about 100,000 or more flows). The results show that the method achieves an accuracy of 96.06 \% and is superior to the latest method ( 3.6 \% F1 score).

4. In view of the continuous explosive growth of mobile applications, and the old model has the problem of disaster forgetting, based on the AIBMF model, a multi-view HTTPS mobile application identification method based on incremental learning--IncreAIBMF is proposed. One of the major drawbacks of proven encrypted traffic identification methods is that new applications continue to suffer from \textit {catastrophic forgetting}, and overall performance can drop dramatically as new applications are gradually added for training. This is because the current model requires the entire dataset, including all samples from the old and new classes, to update the model. As the number of applications grows, the update requirements become unsustainable. To solve this problem, we propose the \emph{IncreAIBMF} framework, which uses new application data and a small sample set of the old application sample to learn step by step. The key idea behind \emph{IncreAIBMF} is an incremental learning framework that incorporates new application recognition capabilities by combining the cross-distillation loss. This learning framework not only learns new application categories, but also retains the old ones Prior knowledge of application categories. The experimental results show that \emph {IncreAIBMF} in real HTTPS traffic data containing 50 mobile applications, respectively, achieves a macro accuracy of 87.3 \%, an F1 score of 87.8 \%, and a macro call of 88.9 \%.

\KEYWORDS{HTTPS Traffic, mobile application identification, multi-view features, deep learning, incremental learning. }% 英文关键词
%---------------------------------------------------------------------------%
