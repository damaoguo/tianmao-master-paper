%---------------------------------------------------------------------------%
%->> Frontmatter
%---------------------------------------------------------------------------%
%-
%-> 生成封面
%-
\maketitle% 生成中文封面
\MAKETITLE% 生成英文封面
%-
%-> 作者声明
%-
\makedeclaration% 生成声明页
%-
%-> 中文摘要
%-
\intobmk\chapter*{摘\quad 要}% 显示在书签但不显示在目录
\setcounter{page}{1}% 开始页码
\pagenumbering{Roman}% 页码符号

本文是中国科学院大学学位论文模板ucasthesis的使用说明文档。主要内容为介绍\LaTeX{}文档类ucasthesis的用法,以及如何使用\LaTeX{}快速高效地撰写学位论文。

\keywords{HTTPS流量识别, 深度学习,迁移学习}% 中文关键词
%-
%-> 英文摘要
%-
\intobmk\chapter*{Abstract}% 显示在书签但不显示在目录

The expanding volume of HTTPS traffic (both legitimate and malicious) creates even more challenges for mobile network security and management. In this work, we propose AIBMF(Application Identification Based on Multi-view Features), a fine-grained approach to classify HTTPS traffic by their application type. The key idea of AIBMF is to combine three kinds of features---payload convolution features, packet size sequence and packet content type sequence. Based on these different view features, a deep learning model (using CNN, embedding and RNN) is constructed for HTTPS traffic identification task. To evaluate the effectiveness of AIBMF, we run a comprehensive set of experiments on a real-world dataset (about 100,000+ flows), which shows that our approach achieves 96.06\% accuracy and outperforms the state-of-the-art method (3.6\% on F1 score). 

\KEYWORDS{HTTPS, mobile application, CNN, embedding, RNN. }% 英文关键词
%---------------------------------------------------------------------------%
