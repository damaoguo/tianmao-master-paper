%---------------------------------------------------------------------------%
%->> Frontmatter
%---------------------------------------------------------------------------%
%-
%-> 生成封面
%-
\maketitle% 生成中文封面
\MAKETITLE% 生成英文封面
%-
%-> 作者声明
%-
\makedeclaration% 生成声明页
%-
%-> 中文摘要
%-
\intobmk\chapter*{摘\quad 要}% 显示在书签但不显示在目录
\setcounter{page}{1}% 开始页码
\pagenumbering{Roman}% 页码符号
随着移动互联网的快速发展,人们生活和工作已经离不开移动只能设备,相应的移动应用程序爆炸性增长,为了智能化运营管理、保证通信质量、净化网络环境,维护网络安全,针对移动应用网络流量的精细化分类已经成为网络安全领域不可或缺的部分。出于通信安全和保护隐私的目的,越来越多的移动应用使用HTTPS协议作为默认的通信协议,HTTPS协议通过加密技术在客户端和服务端建立安全信道,避免通信数据再传输过程中被非法监听、劫持和篡改。HTTPS的广泛使用给传统的流量分析技术带来巨大的挑战,基于负载流量和统计特征的识别方法无法有效完成HTTPS加密流量的精细化分类任务,针对当前移动应用HTTPS流量精细化识别存在的特征建模难,准确率低,鲁棒性差的问题,本文的研究工作将围绕着以上的问题进行展开,具体的研究工作包括:
(1)深入对HTTPS协议的研究,参考RFC文件,近些年来,深度学习在图像,自然语言处理,语音等领域取得了优异的成果,这也为移动应用网络流量识别技术带来了新的契机。
(2)针对当前公共的流量数据集粗糙的问题,构建数据集。
(3)针对当前HTTPS应用特征建模难的问题,提出一种基于深度学习的多视图HTTPS移动应用识别方法---AIBMF。
(4)针对移动应用持续爆炸性增长,而旧模型存在灾难遗忘的问题,在AIBMF模型的基础上,提出一种基于增量学习的多视图HTTPS移动应用识别方法---IncreAIBMF。作为网络安全和管理领域的基本技术,移动应用程序标识面临着一个关键问题,即“加密流量”。经验证的加密流量识别方法具有一个主要缺点,那就是新应用程序继续遭受\textit{灾难性遗忘}的困扰,当逐步增加新的应用程序类别进行培训时,整体性能会急剧下降。这是由于当前模型需要整个数据集(包括来自旧类和新类的所有样本)来更新模型。随着应用程序数量的增长,更新要求变得不可持续,为了解决这个问题,我们建议使用\emph{IncreAIBMF}框架,以使用新的应用程序数据以及仅对应于旧应用程序样本的一小样本集逐步学习深度神经网络。\emph{IncreAIBMF}背后的关键思想是一个增量学习框架,它通过结合交叉蒸馏的损失而拥有新的应用程序识别能力,该学习框架不仅可以学习新的应用程序类别,而且可以保留与旧的应用程序类别相对应的先前知识。我们的实验结果表明,\emph{IncreAIBMF}在包含50个移动应用程序的真实世界跟踪中,分别在宏精度上达到87.3\%,在F1得分上达到87.8\%,在宏调用上达到88.9\%。预测,以及对应用程序类的规模很稳定。此外,\emph{IncreAIBMF}的基本变体,AIBMF在识别性能方面优于最新方法。






HTTPS流量(合法的和恶意的)数量的不断增长对移动网络的安全性和管理提出了更大的挑战。


在这项工作中,我们提出了AIBMF(基于多视图特征的应用程序标识),这是一种按应用程序类型对HTTPS流量进行分类的细粒度方法。AIBMF的关键思想是结合三种功能-有效载荷卷积功能,数据包大小序列和数据包内容类型序列。 基于这些不同的视图特征,构建了用于HTTPS流量识别任务的深度学习模型(使用CNN,嵌入和RNN)。为了评估AIBMF的有效性,我们在一个真实的数据集(大约100,000个以上的流)上进行了一组全面的实验,这表明我们的方法达到了96.06\% 的准确度,并且优于最新方法(3.6 \%F1分数的)。






\keywords{HTTPS流量识别, 深度学习,迁移学习}% 中文关键词
%-
%-> 英文摘要
%-
\intobmk\chapter*{Abstract}% 显示在书签但不显示在目录

The expanding volume of HTTPS traffic (both legitimate and malicious) creates even more challenges for mobile network security and management. In this work, we propose AIBMF(Application Identification Based on Multi-view Features), a fine-grained approach to classify HTTPS traffic by their application type. The key idea of AIBMF is to combine three kinds of features---payload convolution features, packet size sequence and packet content type sequence. Based on these different view features, a deep learning model (using CNN, embedding and RNN) is constructed for HTTPS traffic identification task. To evaluate the effectiveness of AIBMF, we run a comprehensive set of experiments on a real-world dataset (about 100,000+ flows), which shows that our approach achieves 96.06\% accuracy and outperforms the state-of-the-art method (3.6\% on F1 score). 

Mobile application identification, as the fundamental technique in the field of network security and management, suffers from a critical problem, namely 'encrypted traffic’. The proven methods for encrypted traffic identification have a major drawback, which is new come applications continue to suffer from \textit{catastrophic forgetting}, a dramatic decrease in overall performance when training with new app classes added incrementally. This is due to the current model requiring the entire dataset, consisting of all the samples from the old and the new classes, to update the model. The updating requirement becomes easily unsustainable as the number of apps grows, To address the issue, we propose \emph{IncreAIBMF} framework to learn deep neural networks incrementally, using new apps data and only a small exemplar set corresponding to samples from the old apps. The key idea behind \emph{IncreAIBMF} is an incremental learning framework which possesses new application identification ability by incorporating the cross-distilled loss, which can not only learn the new app classes and also retain the previous knowledge corresponding to the old app classes. Our experiment results show that \emph{IncreAIBMF} achieves  87.3\% on Macro Precision, 87.8\% on F1 Score and 88.9\% on Macro Recall, respectively, on the real-world traces that consists of 50 mobile applications, supports the early prediction, and
is robust to the scale of the app classes. Besides, the basic variant of \emph{IncreAIBMF}, AIBMF is superior to the state-of-the-art methods in terms of identification performance.


\KEYWORDS{HTTPS, mobile application, CNN, embedding, RNN. }% 英文关键词
%---------------------------------------------------------------------------%
