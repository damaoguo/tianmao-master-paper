%---------------------------------------------------------------------------%
%->> Frontmatter
%---------------------------------------------------------------------------%
%-
%-> 生成封面
%-
\maketitle% 生成中文封面
\MAKETITLE% 生成英文封面
%-
%-> 作者声明
%-
\makedeclaration% 生成声明页
%-
%-> 中文摘要
%-
\intobmk\chapter*{摘\quad 要}% 显示在书签但不显示在目录
\setcounter{page}{1}% 开始页码
\pagenumbering{Roman}% 页码符号
随着移动互联网的快速发展和移动应用市场的繁荣,移动互联网流量呈现爆炸性增长。如何对这些流量进行有效监管成为网络管理者和服务商亟需面临解决的挑战,也是当前移动应用流量分类领域的研究热点。当前,出于通信安全和隐私保护的目的,大量移动应用采用HTTPS加密技术进行数据承载,这使得基于负载内容签名的传统流量识别方法失效。因此,本文围绕基于HTTPS流量的移动应用识别这一细粒度流量识别问题展开深入研究,主要工作和创新点如下:

1. 针对公开流量数据集标记粗糙以及移动应用流量采集效率低下的问题,提出了一种自动化移动应用 HTTPS 流量的标记方法。该方法借助爬虫技术和软件自动化测试技术,实现了对大规模移动应用的爬取、安装、运行、流量生成和采集标记等功能,构建了一个覆盖11种应用类别、涉及50个移动应用、共计约20多万条HTTPS加密网络数据流的有标记数据集,为后续开展移动应用识别方法的研究和验证评估工作奠定了基础。

2. 针对现有基于HTTPS流量的移动应用识别技术存在识别粒度粗和鲁棒性差的问题,提出一种基于多视图特征的移动应用识别方法。该方法对网络数据流分别从数据包负载字节、数据包大小以及数据包内容类型三种视图角度提取序列特征,并针对性地采用卷积神经网络和循环神经网络进行高阶特征学习和融合,最后采用全连接神经网络实现对HTTPS网络流量的应用识别任务。实验结果表明,该方法达到了91.8\%宏精度、91.3\%的宏$F_1$值以及90.9\%的宏召回率。

3. 针对已构建好的原有移动应用识别模型在面临移动应用持续性动态增加时存在灾难遗忘和模型构建耗能(所需训练时间和存储空间)增长的问题,提出一种基于增量学习的新增移动应用识别方法。该方法在原有模型的基础上,通过使用新增移动应用的流量样本以及原有移动应用的代表性流量样本组成训练集,并引入交叉精馏损失函数进行模型训练,使得模型在保留对原有移动应用的识别能力的同时具备对新增移动应用的识别能力。实验结果表明,该方法通过在对20个移动应用识别的原有模型基础上,通过逐步增加新的移动应用进行增量学习,在最终构建的对50个移动应用的识别模型上分别获得87.3\%的宏精度、88.9\%的召回率以及87.8\%的F1值。

综上所述,本文研究了基于HTTPS流量的移动应用识别技术,通过搭建大规模自动化移动应用流量采集标记平台构建有标记的移动应用HTTPS加密流量样本,通过融合网络流数据包负载字节序列、数据包大小序列、数据包内容类型序列三种特征,并借助深度学习技术进行特征学习和识别模型构建,实现了精细化的移动应用识别任务,并借助增量学习技术一定程度上解决了对新增移动应用的识别问题。


\keywords{HTTPS流量,移动应用识别, 多视图特征,深度学习,增量学习}% 中文关键词
%-
%-> 英文摘要
%-
\intobmk\chapter*{Abstract}% 显示在书签但不显示在目录
With the rapid development of the mobile Internet and the boom of the mobile application market, mobile Internet traffic has exploded. How to effectively monitor these flows has become a challenge that network managers and service providers need to face urgently, and it is also a research hotspot in the field of current mobile application traffic classification. Currently, for the purpose of communication security and privacy protection, a large number of mobile applications use HTTPS encryption technology to carry data, which makes traditional traffic identification methods based on payload content signature invalid. Therefore, this article focuses on the fine-grained traffic identification problem of mobile application identification based on HTTPS traffic. The main work and innovations are as follows:

1. Aiming at the problem of rough tagging of public traffic data sets and inefficient collection of mobile application traffic, a method for automating HTTPS traffic marking for mobile applications is proposed. With the help of crawler technology and software automatic test technology, this method realizes the functions of crawling, installing, running, generating traffic and collecting tags for large-scale mobile applications. This method constructs a labeled data set covering 11 application categories, involving 50 mobile applications, and a total of about 200,000 HTTPS encrypted network data streams. This dataset laid the foundation for the subsequent research and verification and evaluation of mobile application identification methods.

2. Aiming at the problems of existing coarse recognition granularity and poor robustness of existing mobile application recognition technologies based on HTTPS traffic, a mobile application recognition method based on multi-view features is proposed. This method extracts sequence features for the network data stream from three perspectives: data packet payload byte, data packet size, and data packet content type, and uses convolutional neural network and recurrent neural network for high-level feature learning and fusion. Finally, a fully connected neural network is used to implement the application identification task of HTTPS network traffic. Experimental results show that the method achieves 91.8\% Macro Precision, 90.9\% Macro Recall and 91.3 \% Macro $ F_1 $.

3. Aiming at the problems of the existing mobile application recognition model that faced with the continuous dynamic increase of mobile applications, catastrophic forgetting and model building energy consumption (required training time and storage space) increase, this paper proposes a method based on incremental learning for new mobile app identification method. Based on the original model, the method uses the new mobile application's traffic samples and representative traffic samples of the original mobile application to form a training set, and introduces a cross-distillation loss function for model training, so that the model retains the recognition ability of the mobile application ans has the recognition ability for the newly added mobile application. Experimental results show that the method the final recognition model of 50 mobile applications by incrementally adding new mobile applications for incremental learning based on the original model of 20 mobile application recognition obtains 87.3\% Macro Precision, 88.9\% Macro Recall, and 87.8\% Macro $F_1$.

To sum up, this paper studies the mobile application identification technology based on HTTPS traffic. By building a large-scale automated mobile application traffic collection and labellinging platform, a labeled mobile application HTTPS encrypted traffic dataset is constructed. A refined mobile application recognition method is proposed by merging the network stream data packet load byte sequence, packet size sequence and data packet content type sequence and using deep learning to extract features. With the help of incremental learning technology, the identification of new mobile applications can be solved.

\KEYWORDS{HTTPS Traffic, mobile application identification, multi-view features, deep learning, Incremental Learning. }% 英文关键词
%---------------------------------------------------------------------------%
