\chapter{基于HTTPS流量的HTTPS流量应用多视图特征识别方法的研究}\label{chap:AIBMF}


\section{引入}

\section{多视图特征}
\subsection{payload部分}

\subsection{parameter部分}
\citep{wiana2019}

\subsection{packet size部分}

\subsection{神经网络结构}
% =================================
% Fig. 03: Model architecture
% =================================
\begin{figure}[!htbp]
	\centering
	\includegraphics[width=0.80\textwidth]{AIBMF-model.pdf}
	\bicaption{AIBMF网络结构}{AIBMF Architecture}
	\label{fig:AIBMF_Architecture}
\end{figure}




\section{多视图识别模型评估}
\subsection{评估标准}
% =================================
% EQUATION:
% =================================
\begin{equation}
ACC=\frac{TP+TN}{TP+TN+FP+FN}	
\end{equation}


% =================================
% EQUATION:
% =================================
\begin{subequations}
	\begin{equation}
	P=\frac{TP}{TP+FP}
	\end{equation}
	
	\begin{equation}
	Macro\ Precision=\frac{1}{20}\sum_{i=1}^{20}P_i
	\end{equation}
	
\end{subequations}

% =================================
% EQUATION:08
% =================================
\begin{subequations}
	\begin{equation}
	R=\frac{TP}{TP+FN}	
	\end{equation}
	
	\begin{equation}
	Macro\ Recall=\frac{1}{20}\sum_{i=1}^{20}R_i
	\end{equation}
\end{subequations}


% =================================
% EQUATION:09
% =================================
\begin{subequations}
	\begin{equation}
	F_1=2\frac{P\times R}{P+R}
	\end{equation}
	
	\begin{equation}
	Macro\ F_1=\frac{1}{20}\sum_{i=1}^{20}F_{1i}
	\end{equation}
\end{subequations}




\subsection{超参数的确定}
\subsubsection{单个样本packet数目确定}
% =================================
% Fig. 03: Model architecture
% =================================
\begin{figure}[!htbp]
	\centering
	\includegraphics[width=0.80\textwidth]{pkts_count.png}
	\bicaption{单个样本packet数目}{packet count comparation}
	\label{fig:packet count}
\end{figure}



\subsubsection{单个packet的payload字节数确定}

% =================================
% Fig. 03: Model architecture
% =================================
\begin{figure}[!htbp]
	\centering
	\includegraphics[width=0.80\textwidth]{pkt_size_acc.png}
	\bicaption{packet size}{view comparation}
	\label{fig:packet size}
\end{figure}



\subsection{不同视图选择的评估}
% =================================
% Fig. 03: Model architecture
% =================================
\begin{figure}[!htbp]
	\centering
	\includegraphics[width=0.80\textwidth]{acc_loss.png}
	\bicaption{不同视图比较}{view comparation}
	\label{fig:view comparation}
\end{figure}



\subsection{不同深度学习模型的评估}

% =================================
% Fig. 03: Model architecture
% =================================
\begin{figure}[!htbp]
	\centering
	\includegraphics[width=0.80\textwidth]{comparation.png}
	\bicaption{不同深度学习模型的评估}{model comparation}
	\label{fig:AIBMF_comparation}
\end{figure}



\subsection{传统方法和研究方法的对比}
% ===========
% Table 01:Statistics
% ===========
\begin{table}
	\caption{HTTPS Statistics}
	\setlength{\tabcolsep}{7mm}
	\begin{center}
		\begin{tabular}{c}
			\hline
			\textbf{Statistics}\\
			\hline
			\hline
			Packet counts\\
			\hline
			TTL max\\
			TTL min\\
			TTL mean\\
			TTL median\\
			TTL var\\
			\hline
			packet\_length max\\
			packet\_length min\\
			packet\_length mean\\
			packet\_length median\\
			packet\_length var\\
			\hline	
			window max\\
			window min\\
			window mean\\
			window median\\
			window var\\
			\hline
			session\_id\_length max\\
			\hline	
			client\_extensions\_length max\\
			client\_extensions\_length min\\
			client\_extensions\_length mean\\
			client\_extensions\_length median\\
			client\_extensions\_length var\\
			\hline
			client\_ciphers counts\\
			\hline
			server\_cipher\\
			\hline
		\end{tabular}
		\label{tab1}
	\end{center}
\end{table}

\begin{table}
	\caption{traffic identification based on statistical features only}
	\begin{center}
		\begin{tabular}{|c||c||c||c|}
			\hline
			\textbf{Machine learning algorithm} & \textbf{Precision} & \textbf{Recall} & \textbf{F1} \\
			\hline
			\hline
			Random Forest & 0.8805 & 0.8172 & 0.8407 \\
			\hline
			SVM-RBF & 0.9214 & 0.5754 & 0.6676 \\
			\hline
			DNN & 0.7362 & 0.6293 & 0.6562 \\
			\hline
			\textbf{AIBMF} & \textbf{0.918} & \textbf{0.909} & \textbf{0.913} \\
			\hline
		\end{tabular}
		\label{tab4}
	\end{center}
\end{table}



\section{在混合模式下的效果}
% =================================
% Fig. 03: Model architecture
% =================================
\begin{figure}[!htbp]
	\centering
	\includegraphics[width=0.80\textwidth]{AIBMF-confusematrix.png}
	\bicaption{AIBMF识别混淆矩阵}{AIBMF classification confusematrix}
	\label{fig:AIBMF_confusematrix}
\end{figure}


\section{小结}