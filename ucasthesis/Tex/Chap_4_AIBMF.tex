\chapter{基于HTTPS流量的HTTPS流量应用多视图特征识别方法的研究}\label{chap:AIBMF}


\section{引入}

\section{多视图特征}
\subsection{payload部分}

\subsection{parameter部分}
\citep{wiana2019}

\subsection{packet size部分}

\subsection{神经网络结构}
% =================================
% Fig. 03: Model architecture
% =================================
\begin{figure}[!htbp]
	\centering
	\includegraphics[width=0.80\textwidth]{AIBMF-model.pdf}
	\bicaption{AIBMF网络结构}{AIBMF Architecture}
	\label{fig:AIBMF_Architecture}
\end{figure}




\section{多视图识别模型评估}
\subsection{评估标准}
In order to make a reasonable and effective quantitative evaluation of the identification performance of AIBMF on the traffic data, this paper introduces some basic metrics: True positives (TP), False negatives (FN), True negatives (TN), and False positives (FP). TP is the number of flows classified to app $i$ exactly belong to app $i$. FN is the Number of flows classified  to app $i$ exactly belong to app $j$. TN is the Number of flows classified  to app $j$ exactly belong to app $i$. FP is the number of flows classified  to app $j$ exactly belong to app $j$.


In the training phase, the \emph{Accuracy(ACC)} is used to indicate the improvement of the model identification ability, and the performance of the comprehensive display model in identifying HTTPS traffic is improved. The accuracy rate is the proportion of all the correct sample sizes to the training data during the iterative training, which is calculated according to formula (7). 


% ===========
% EQUATION:06
% ===========
\begin{equation}
ACC=\frac{TP+TN}{TP+TN+FP+FN}	
\end{equation}

In the test phase, \emph{Precision(P)} that shows how many flows predicted to app $i$ are actual app $i$, \emph{Recall(R)} that shows the percentage of flows that are correctly predicted versus all flows that belong to app $i$, and the comprehensive evaluation index \emph{F1} value for HTTPS traffic are used as the evaluation criteria. \emph{P} is calculated according to formula (8a); \emph{R} is calculated according to formula (9a); the F1 value is a comprehensive evaluation index that combines two indicators, and is calculated according to formula (10a). To evaluate our method on all 20 apps, we introduce the average of $P$, $R$, $F_1$ --- $Macro\ Precision$, $Macro\ Recall$, $Macro\ F_1$ as the overall metrics according to formula (8b), (9b), (10b). 


% ===========
% EQUATION:07
\begin{subequations}
	\begin{equation}
	P=\frac{TP}{TP+FP}
	\end{equation}
	
	\begin{equation}
	Macro\ Precision=\frac{1}{20}\sum_{i=1}^{20}P_i
	\end{equation}
	
\end{subequations}

% ===========
% EQUATION:08
% ===========
\begin{subequations}
	\begin{equation}
	R=\frac{TP}{TP+FN}	
	\end{equation}
	
	\begin{equation}
	Macro\ Recall=\frac{1}{20}\sum_{i=1}^{20}R_i
	\end{equation}
\end{subequations}


% ===========
% EQUATION:09
% ===========
\begin{subequations}
	\begin{equation}
	F_1=2\frac{P\times R}{P+R}
	\end{equation}
	
	\begin{equation}
	Macro\ F_1=\frac{1}{20}\sum_{i=1}^{20}F_{1i}
	\end{equation}
\end{subequations}



\subsection{超参数的确定}
\subsubsection{单个样本packet数目确定}
% =================================
% Fig. 03: Model architecture
% =================================
\begin{figure}[!htbp]
	\centering
	\includegraphics[width=0.80\textwidth]{pkts_count.png}
	\bicaption{单个样本packet数目}{packet count comparation}
	\label{fig:packet count}
\end{figure}



\subsubsection{单个packet的payload字节数确定}

% =================================
% Fig. 03: Model architecture
% =================================
\begin{figure}[!htbp]
	\centering
	\includegraphics[width=0.80\textwidth]{pkt_size_acc.png}
	\bicaption{packet size}{view comparation}
	\label{fig:packet size}
\end{figure}



\subsection{不同视图选择的评估}
% =================================
% Fig. 03: Model architecture
% =================================
\begin{figure}[!htbp]
	\centering
	\includegraphics[width=0.80\textwidth]{acc_loss.png}
	\bicaption{不同视图比较}{view comparation}
	\label{fig:view comparation}
\end{figure}



\subsection{不同深度学习模型的评估}

% =================================
% Fig. 03: Model architecture
% =================================
\begin{figure}[!htbp]
	\centering
	\includegraphics[width=0.80\textwidth]{comparation.png}
	\bicaption{不同深度学习模型的评估}{model comparation}
	\label{fig:AIBMF_comparation}
\end{figure}



\subsection{传统方法和研究方法的对比}
% ===========
% Table 01:Statistics
% ===========
\begin{table}
	\caption{HTTPS Statistics}
	\setlength{\tabcolsep}{7mm}
	\begin{center}
		\begin{tabular}{| c | c | c |}
			\hline
			\textbf{统计特征} & \textbf{含义} & \textbf{是否特有?}\\
			\hline
			\hline
			Packet counts & Packet数量& 否\\
			\hline
			TTL max & TTL最大值 & 否\\
			\hline
			TTL min & TTL最小值 & 否\\
			\hline
			TTL mean & TTL平均值 & 否\\
			\hline
			TTL median & TTL中位数 & 否\\
			\hline
			TTL var & TTL方差 & 否\\
			\hline
			packet\_length max & packet长度最大值 & 否\\
			\hline
			packet\_length min & packet长度最小值 & 否\\
			\hline
			packet\_length mean & packet长度平均值 & 否\\
			\hline
			packet\_length median & packet长度中位数 & 否\\
			\hline
			packet\_length var & packet长度方差 & 否\\
			\hline	
			window max & 窗口最大值 & 否\\
			\hline
			window min & 窗口最小值 & 否\\
			\hline
			window mean & 窗口平均值 & 否\\
			\hline
			window median & 窗口中位数 & 否\\
			\hline
			window var & 窗口方差 & 否\\
			\hline
			session\_id\_length max & session id长度 & 否\\
			\hline	
			client\_extensions\_length max & client extensions长度最大值 & 是\\
			\hline
			client\_extensions\_length min & client extensions长度最小值 & 是\\
			\hline
			client\_extensions\_length mean & client extensions长度平均值 & 是\\
			\hline
			client\_extensions\_length median & client extensions长度中位数 & 是\\
			\hline
			client\_extensions\_length var & client extensions长度方差 & 是\\
			\hline
			client\_ciphers counts & 客户端加密组件种类 & 是\\
			\hline
			server\_cipher & 服务端加密组建类型 & 是\\
			\hline
		\end{tabular}
		\label{tab1}
	\end{center}
\end{table}

\begin{table}
	\caption{traffic identification based on statistical features only}
	\begin{center}
		\begin{tabular}{|c||c||c||c|}
			\hline
			\textbf{Machine learning algorithm} & \textbf{Precision} & \textbf{Recall} & \textbf{F1} \\
			\hline
			\hline
			Random Forest & 0.8805 & 0.8172 & 0.8407 \\
			\hline
			SVM-RBF & 0.9214 & 0.5754 & 0.6676 \\
			\hline
			DNN & 0.7362 & 0.6293 & 0.6562 \\
			\hline
			\textbf{AIBMF} & \textbf{0.918} & \textbf{0.909} & \textbf{0.913} \\
			\hline
		\end{tabular}
		\label{tab4}
	\end{center}
\end{table}



\section{在混合模式下的效果}
% =================================
% Fig. 03: Model architecture
% =================================
\begin{figure}[!htbp]
	\centering
	\includegraphics[width=0.80\textwidth]{AIBMF-confusematrix.png}
	\bicaption{AIBMF识别混淆矩阵}{AIBMF classification confusematrix}
	\label{fig:AIBMF_confusematrix}
\end{figure}


\section{小结}