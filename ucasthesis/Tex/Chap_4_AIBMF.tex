\chapter{基于HTTPS流量的HTTPS流量应用多视图特征识别方法的研究}\label{chap:AIBMF}

\section{引入}
Our insights of \emph{AIBMF} into encrypted application identification is three-fold:
%
\emph{(i)} SSL/TLS protocol is essentially a language of communication between mobile app client and server, the specific manifestation of this language is network flow, \emph{i.e.}, a series of packets;
%
\emph{(ii)} Because protocol state machine is an overall abstract description of the network flows, application fingerprints can be constructed from different protocol state information;
%
\emph{(iii)} Although SSL/TLS hides the payload of network flows, side-channel protocol state information is still leaked from encrypted connections. 

\section{多视图特征}
Three views are incorporated in our model which are named as packet size (PS) view, packet content type (CT) view and packet payload byte (PB) view respectively. 
%
This framework includes three modules: preprocessing, feature learning, and identification.


This module first organizes raw network traffic into flows (\emph{i.e.}, the sample objects to be classified). 
%
The collection of the whole flows can be indicated as $D = \{{\{f_i\}}_{i=1}^{i=N}\}_{j=1}^{j=C}$ where $f_i$ means the $i^{th}$ flow. 
%
% In this paper, the flow objects defined by the $\{SRC_{IP}, SRC_{port}, DEST_{IP}, DEST_{port},$ $Transport Protocol\}$ tuple.
%
% and an inactivity timeout, with a default value of 60s.
%
As we already have the flow objects, we can turn to the numerical vector representation of flow.
%
\emph{(i)} For PS view, we form a $16$ dimensional vector $v_{ps}$ by computing the packet size of the first 16 packets (padded with -1).
%
\emph{(ii)} For CT view, we form a $16$ dimensional vector $v_{ct}$ by extacting the value content type field the first 16 packets (padded with -1), and it ranges ranges from 0 to 255.
%
\emph{(iii)} For PB view, we form a $16\times64 = 1024$  dimensional vector $v_{pb}$ by regarding payload byte value as the corresponding ASCII code, and further convert into a integer of 0-255 (padded with -1). 
\begin{itemize}

	\item \textbf{Feature learning from PB view:} 
	%
	We use a 1D-CNN network architecture to process the input numerical vector $v_{pb}$.
	%
	In the 1D-CNN, two-layer convolutional are used, the number of convolution kernels used in each layer is 64, and the size of the convolution kernel is $1 \times 3$, max pooling layer used. 
	%
	Therefore, raw payload byte sequence is transformed into a 64-dimensional vector $\alpha$.
	%
	\begin{equation}
		\alpha = \mathit{1\!D\!-\!C\!N\!N}(v_{pb})
	\end{equation}
	%
	% We let $v_{i:i+j^{pb}}$ refer to the concatenation of ASCII code of payload $v_i,v_{i+1},...,v_{i+j}$.
	% %
	% The convolution kernel is defined in the real space. 
	% %
	% Here, $h=3,k=1$, each feature $c_i$ is calculated from a sequence window $v_{i:i+h-1}^{pb}$:
	% %
	% 	\begin{equation}
	% 		c_i=f(w \cdot v_{i:i+h-1}+b)
	% 	\end{equation}
	% %
	% Here is a bias term and $f$ is a nonlinear function. The convolution kernel is applied to all windows $\{x_{1:h}, x_{2:h+1},...,x_{n-h+1:n}\}$ of the packet payload and packet size sequence in the flow to produce a feature map.
	% %
	% Here $n=1024$ is the input length. Then we apply max pooling on the result of the convolution calculation and take the maximum  as the result of the convolution calculation.
	% %
	% After two-layers convolution, payload is transformed into a 64-long vector.
	%
	\item \textbf{Feature learning from PS view:}
	%
	The vector $v_{ps}$ is processed by 1D-CNN the same as payload byte sequence, the differences are that the input vector is packet size sequence and the input length is 16 here.
	%
	After 1D-CNN feature learning, packet size sequence is transformed into a 64-dimensional vector $\beta$.
	%
	\begin{equation}
		\beta = \mathit{1D\!-\!C\!N\!N}(v_{ps})
	\end{equation}
	%
	\item \textbf{Feature learning from CT view:} 
	%
	The content type is originally a discrete variable represented by a 257-dimensional one-hot vector which is very sparse, so we introduce the embedding operation which maps the sparse representation to a 16-dimensional real vector which incorporates contextual information about the content type.
	%
	% Embedding mathetextbfcally represents a mapping: $f:X \to Y$, which satisfies two properties: injective and structure-preserving.
	%
	The Embedding layer needs to learn a transform matrix whose shape is $ (vocab\_size, embedding\_dim) $.
	%
	The embedding layer is basically a matrix which can be considered a transformation from discrete and sparse 1-hot-vector into a continuous and dense latent space. 
	%
	Formally, 
	%
	\begin{equation}
		\gamma = \mathit{R\!N\!N}(Embedding(v_{ct}))
	\end{equation}
	%
\end{itemize}

\subsection{payload部分}

\subsection{parameter部分}
\citep{wiana2019}

\subsection{packet size部分}

\subsection{神经网络结构}
% =================================
% Fig. 03: Model architecture
% =================================
\begin{figure}[!htbp]
	\centering
	\includegraphics[width=0.80\textwidth]{AIBMF-model.pdf}
	\bicaption{AIBMF网络结构}{AIBMF Architecture}
	\label{fig:AIBMF_Architecture}
\end{figure}

\subsubsection{Identification Module:}
%
Three different abstract features are concatenated together to form a 192-dimensional vector
% 
\begin{equation}
	\delta = \alpha \oplus \beta \oplus \gamma
\end{equation}
%
which is then input to the fully connected layer. 
%
The output of the fully connected layer, \emph{i.e.}, $x$, will be used as the input of the softmax regression to identify the flow. 
%
% A flow can be identified into $C$ classes, where $M=K$ denotes all type of apps. 
%
Three different abstract features jointly optimize the same category target, jointly calculate the multi-class cross entropy loss, and update the parameters of the three neural networks through the back propagation algorithm\cite{rumelhart1986learning}.
%
Mathematically, the probability that an output prediction $Y$ is app $a_j$, is determined by: 
\begin{equation}
p(Y=j|x,W,b)=\mathit{softmax_j}(\emph{\textbf{W}}x + \emph{\textbf{b}})
% =\frac{e^{w_i^TZ+b_i}}{\sum_je^{w_j^TZ+b_j}}
\end{equation}
where $W$ is a weight matrix between the fully connected layer and the softmax layer, and the $b$ is the bais vector.
% 
Then the model's prediction $y_{pd} $ is the class whose probability is maximal:
%
\begin{equation}
%
y_{pd}=argmax(p(Y=j|x,W,b)),\forall j \in \{1,2,3,...,C\}
\end{equation} 
%
Our neural network selects the cross entropy \cite{Boer2005A} as the loss function. 
%


\section{多视图识别模型评估}
\subsection{评估标准}
In order to make a reasonable and effective quantitative evaluation of the identification performance of AIBMF on the traffic data, this paper introduces some basic metrics: True positives (TP), False negatives (FN), True negatives (TN), and False positives (FP). TP is the number of flows classified to app $i$ exactly belong to app $i$. FN is the Number of flows classified  to app $i$ exactly belong to app $j$. TN is the Number of flows classified  to app $j$ exactly belong to app $i$. FP is the number of flows classified  to app $j$ exactly belong to app $j$.


In the training phase, the \emph{Accuracy(ACC)} is used to indicate the improvement of the model identification ability, and the performance of the comprehensive display model in identifying HTTPS traffic is improved. The accuracy rate is the proportion of all the correct sample sizes to the training data during the iterative training, which is calculated according to formula (7). 


% ===========
% EQUATION:06
% ===========
\begin{equation}
ACC=\frac{TP+TN}{TP+TN+FP+FN}	
\end{equation}

In the test phase, \emph{Precision(P)} that shows how many flows predicted to app $i$ are actual app $i$, \emph{Recall(R)} that shows the percentage of flows that are correctly predicted versus all flows that belong to app $i$, and the comprehensive evaluation index \emph{F1} value for HTTPS traffic are used as the evaluation criteria. \emph{P} is calculated according to formula (8a); \emph{R} is calculated according to formula (9a); the F1 value is a comprehensive evaluation index that combines two indicators, and is calculated according to formula (10a). To evaluate our method on all 20 apps, we introduce the average of $P$, $R$, $F_1$ --- $Macro\ Precision$, $Macro\ Recall$, $Macro\ F_1$ as the overall metrics according to formula (8b), (9b), (10b). 


% ===========
% EQUATION:07
\begin{subequations}
	\begin{equation}
	P=\frac{TP}{TP+FP}
	\end{equation}
	
	\begin{equation}
	Macro\ Precision=\frac{1}{20}\sum_{i=1}^{20}P_i
	\end{equation}
	
\end{subequations}

% ===========
% EQUATION:08
% ===========
\begin{subequations}
	\begin{equation}
	R=\frac{TP}{TP+FN}	
	\end{equation}
	
	\begin{equation}
	Macro\ Recall=\frac{1}{20}\sum_{i=1}^{20}R_i
	\end{equation}
\end{subequations}


% ===========
% EQUATION:09
% ===========
\begin{subequations}
	\begin{equation}
	F_1=2\frac{P\times R}{P+R}
	\end{equation}
	
	\begin{equation}
	Macro\ F_1=\frac{1}{20}\sum_{i=1}^{20}F_{1i}
	\end{equation}
\end{subequations}



\subsection{超参数的确定}
\subsubsection{单个样本packet数目确定}
% =================================
% Fig. 03: Model architecture
% =================================
\begin{figure}[!htbp]
	\centering
	\includegraphics[width=0.80\textwidth]{pkts_count.png}
	\bicaption{单个样本packet数目}{packet count comparation}
	\label{fig:packet count}
\end{figure}



\subsubsection{单个packet的payload字节数确定}

% =================================
% Fig. 03: Model architecture
% =================================
\begin{figure}[!htbp]
	\centering
	\includegraphics[width=0.80\textwidth]{pkt_size_acc.png}
	\bicaption{packet size}{view comparation}
	\label{fig:packet size}
\end{figure}



\subsection{不同视图选择的评估}
% =================================
% Fig. 03: Model architecture
% =================================
\begin{figure}[!htbp]
	\centering
	\includegraphics[width=0.80\textwidth]{acc_loss.png}
	\bicaption{不同视图比较}{view comparation}
	\label{fig:view comparation}
\end{figure}



\subsection{不同深度学习模型的评估}

% =================================
% Fig. 03: Model architecture
% =================================
\begin{figure}[!htbp]
	\centering
	\includegraphics[width=0.80\textwidth]{AIBMF-Other-comparation.eps}
	\bicaption{不同深度学习模型的评估}{model comparation}
	\label{fig:AIBMF_comparation}
\end{figure}



\subsection{传统方法和研究方法的对比}
% ===========
% Table 01:Statistics
% ===========
\begin{table}
	\caption{HTTPS Statistics}
	\setlength{\tabcolsep}{7mm}
	\begin{center}
		\begin{tabular}{| c | c | c |}
			\hline
			\textbf{统计特征} & \textbf{含义} & \textbf{是否特有?}\\
			\hline
			\hline
			Packet counts & Packet数量& 否\\
			\hline
			TTL max & TTL最大值 & 否\\
			\hline
			TTL min & TTL最小值 & 否\\
			\hline
			TTL mean & TTL平均值 & 否\\
			\hline
			TTL median & TTL中位数 & 否\\
			\hline
			TTL var & TTL方差 & 否\\
			\hline
			packet\_length max & packet长度最大值 & 否\\
			\hline
			packet\_length min & packet长度最小值 & 否\\
			\hline
			packet\_length mean & packet长度平均值 & 否\\
			\hline
			packet\_length median & packet长度中位数 & 否\\
			\hline
			packet\_length var & packet长度方差 & 否\\
			\hline	
			window max & 窗口最大值 & 否\\
			\hline
			window min & 窗口最小值 & 否\\
			\hline
			window mean & 窗口平均值 & 否\\
			\hline
			window median & 窗口中位数 & 否\\
			\hline
			window var & 窗口方差 & 否\\
			\hline
			session\_id\_length max & session id长度 & 否\\
			\hline	
			client\_extensions\_length max & client extensions长度最大值 & 是\\
			\hline
			client\_extensions\_length min & client extensions长度最小值 & 是\\
			\hline
			client\_extensions\_length mean & client extensions长度平均值 & 是\\
			\hline
			client\_extensions\_length median & client extensions长度中位数 & 是\\
			\hline
			client\_extensions\_length var & client extensions长度方差 & 是\\
			\hline
			client\_ciphers counts & 客户端加密组件种类 & 是\\
			\hline
			server\_cipher & 服务端加密组建类型 & 是\\
			\hline
		\end{tabular}
		\label{tab1}
	\end{center}
\end{table}

\begin{table}
	\caption{traffic identification based on statistical features only}
	\begin{center}
		\begin{tabular}{|c||c||c||c|}
			\hline
			\textbf{Machine learning algorithm} & \textbf{Precision} & \textbf{Recall} & \textbf{F1} \\
			\hline
			\hline
			Random Forest & 0.8805 & 0.8172 & 0.8407 \\
			\hline
			SVM-RBF & 0.9214 & 0.5754 & 0.6676 \\
			\hline
			DNN & 0.7362 & 0.6293 & 0.6562 \\
			\hline
			\textbf{AIBMF} & \textbf{0.918} & \textbf{0.909} & \textbf{0.913} \\
			\hline
		\end{tabular}
		\label{tab4}
	\end{center}
\end{table}



\section{在混合模式下的效果}
% =================================
% Fig. 03: Model architecture
% =================================
\begin{figure}[!htbp]
	\centering
	\includegraphics[width=0.80\textwidth]{AIBMF-confusematrix.png}
	\bicaption{AIBMF识别混淆矩阵}{AIBMF classification confusematrix}
	\label{fig:AIBMF_confusematrix}
\end{figure}


\section{小结}